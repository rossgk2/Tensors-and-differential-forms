\chapter{Bilinear forms, metric tensors, and coordinates of tensors}

\label{ch::bilinear_forms_metric_tensors}

\section{Bilinear forms}

\begin{defn}
\label{ch::bilinear_forms_metric_tensors::defn::linear_k_form}
    (Linear $k$-form, bilinear form).
    
    Let $V_1, ..., V_k$ be vector spaces over a field $K$. A \textit{linear $k$-form on $V_1, ..., V_k$} is\footnote{Unfortunately, the word ``$k$-form'' without the qualifier ``linear'' is reserved to mean \textit{differential} $k$-form. We have not defined differential $k$-forms yet.} a $k$-linear function ${V_1 \times ... \times V_k \rightarrow K}$.
    
    Let $V$ be a vector space over $K$. A \textit{linear $k$-form on $V$} is a linear $k$-form on $V^{\times k}$.
    
    A \textit{bilinear form on $V_1$ and $V_2$} is a linear $2$-form on $V_1$ and $V_2$, and a bilinear form on $V$ is a linear $2$-form on $V$ and $V$, i.e., a bilinear form on $V$ and $V$.
\end{defn}

\begin{remark}
\label{ch::bilinear_forms_metric_tensors::rmk::linear_k_forms_0_k_tensors}

    (Linear $k$-forms are naturally identified with $(0, k)$ tensors).
    
    A linear $k$-form on $V$ is an element of $\LLLL(V^{\times k} \rightarrow K)$. Recalling Theorem \ref{ch::motivated_intro::thm::four_fundamental_isos}, we have $\LLLL(V^{\times k} \rightarrow K) \cong \LLLL(V^{\otimes k} \rightarrow K) = (V^{\otimes k})^* \cong (V^*)^{\otimes k} = T^0_k(V)$. Therefore a linear $k$-form is naturally identified with a $(0, k)$ tensor.
\end{remark}

\begin{defn}
\label{ch::bilinear_forms_metric_tensors::defn::nondegen_bilinear_form}
    (The musical functions).
    
    Let $V$ and $W$ be finite-dimensional vector spaces. If $B$ is a bilinear form on $V$ and $W$, then there are linear functions $\flat_1:V \rightarrow W^{*}$ and $\flat_2:W \rightarrow V^{*}$ defined\footnote{$B(\vv, \cdot)$ denotes the function $\ww \mapsto B(\vv, \ww)$ and $B(\cdot, \ww)$ denotes the function $\vv \mapsto B(\vv, \ww)$.} by $\flat_1(\vv) := B(\vv, \cdot)$ and $\flat_2(\ww) := B(\cdot, \ww)$. These are called the \textit{musical functions (induced by $B$)}. We denote $\vv^{\flat_1} := \flat_1(\vv)$ and $\ww^{\flat_2} := \flat_2(\ww)$.
\end{defn}

\begin{theorem}
    (The musical isomorphisms).

    Let $V$ and $W$ be finite-dimensional vector spaces and let $B$ be a bilinear form on $V$ and $W$. The musical functions $\flat_1$ and $\flat_2$ are linear isomorphisms iff the following equivalent conditions hold:

    \begin{enumerate}
        \item $\flat_1$ and $\flat_2$ are one-to-one
        \item $B(\vv, \ww) = \mathbf{0} \iff \vv = \mathbf{0} \text{ or } \ww = \mathbf{0} \text{ for all } \vv \in V, \ww \in W$
    \end{enumerate}

    When they are isomorphisms, $\flat_1$ and $\flat_2$ are called the \textit{musical isomorphisms (induced by $B$)}. We denote the inverses of $\flat_1$ and $\flat_2$ by $\sharp_1$ and $\sharp_2$, respectively: $\sharp_1 = \flat_1^{-1}:W^* \rightarrow V$ and $\sharp_2 = \flat_2^{-1}:V^* \rightarrow W$.

    Note that the musical isomorphisms are natural isomorphisms because they do not depend on a choice of basis (see Definition \ref{ch::lin_alg::defn::natural_iso}).
\end{theorem}

\begin{proof}
    \mbox{} \\
    \begin{enumerate}
        \item ($\flat_1$ and $\flat_2 \implies \flat_1$ and $\flat_2$ are linear isomorphisms). Assume $\flat_1:V \rightarrow W^*$ and $\flat_2:W \rightarrow V^*$ are one-to-one. Then $\dim(V) \leq \dim(W^*) = \dim(W)$ and $\dim(W) \leq \dim(V^*) = \dim(V)$, so $\dim(V) = \dim(W)$. Since $\flat_1$ and $\flat_2$ one-to-one linear functions between finite-dimensional vectors paces of the same dimension, then $\flat_1$ and $\flat_2$ are linear isomorphisms (recall Theorem \ref{ch::lin_alg::thm::linear_fn_1-1_iff_onto}).

        ($\flat_1$ and $\flat_2$ are linear isomorphisms $\implies \flat_1$ and $\flat_2$ are one-to-one). This is trivial.
        
        \item We have: $(\flat_1 \text{ is one-to-one}) \iff (\vv^{\flat_1} = \mathbf{0} \implies \vv = \mathbf{0}) \iff ((\vv^{\flat_1}(\ww) = \mathbf{0} \text{ for all $\ww$}) \implies (\vv = \mathbf{0}))$ \\ $\iff B(\vv, \ww) = \mathbf{0} \implies \vv = \mathbf{0}$. Similarly, $(\flat_2 \text{ is one-to-one}) \iff (B(\vv, \ww) = \mathbf{0} \implies \ww = \mathbf{0})$. The fact (2) is equivalent to these two results.
    \end{enumerate}
\end{proof}

The second condition of the previous theorem describes bilinear forms that generate musical isomorphisms and therefore motivates a new definition.

\begin{defn}
    (Nondegenerate bilinear form).
    
    Let $V$ and $W$ be finite-dimensional vector spaces. A bilinear form $B$ on $V$ and $W$ is called \textit{nondegenerate} iff $(B(\vv, \ww) = \mathbf{0} \iff \vv = \mathbf{0} \text{ or } \ww = \mathbf{0}) \text{ for all } \vv \in V, \ww \in W$.

    We saw above that a bilinear form generates musical isomorphisms iff it is nondegenerate.
\end{defn}

\begin{theorem}
\label{ch::bilinear_forms_metric_tensors::induced_bilinear_form_on_duals}
    (Induced bilinear form on the duals).
    
    Let $V$ and $W$ be finite-dimensional vector spaces over a field $K$. If $B$ is a nondegenerate bilinear form on $V$ and $W$, then there is an induced nondegenerate bilinear form $\widetilde{B}:W^* \times V^* \rightarrow K$ on $W^*$ and $V^*$ defined by $\widetilde{B}(\psi, \phi) = B(\psi^{\sharp_1}, \phi^{\sharp_2})$. The bilinear form $\widetilde{B}$ induces natural isomorphisms $\widetilde{\flat}_1:V^* \rightarrow W^{**}$ and $\widetilde{\flat}_2:W^* \rightarrow V^{**}$ defined by $\phi^{\widetilde{\flat}_1}(\psi) = \widetilde{B}(\psi, \phi)$ and $\psi^{\widetilde{\flat}_2}(\phi) = \widetilde{B}(\psi, \phi)$.
\end{theorem}

\begin{proof}
    The proof that $\widetilde{B}$ is actually a nondegenerate bilinear form is left as an exercise.
\end{proof}

\begin{lemma}
     Let $V$ and $W$ be finite-dimensional vector spaces with bases $E = \{\ee_1, ..., \ee_n\}$ and $F = \{\ff_1, ..., \ff_m\}$, and let $E^* = \{\phi^{\ee_1}, ..., \phi^{\ee_n}\}$ and $F^* = \{\psi^{\ff_1}, ..., \psi^{\ff_m}\}$ be the bases for $V^*$ and $W^*$ induced by $E$ and $F$. Let $B$ be a nondegenerate bilinear form on $V$ and $W$, and let $\widetilde{B}$ be the induced bilinear form on $W^*$ and $V^*$. Then 

    \begin{align*}
        B(\vv, \ww) &= [\vv]_E^\top \BB [\ww]_F = [\ww]_F^\top \BB^\top [\vv]_E, \text{ where $\BB := \Big(B(\ee_i, \ee_j)\Big)$}
        \\
        \widetilde{B}(\psi, \phi) &= [\psi]_{F^*}^\top \widetilde{\BB} [\phi]_{E^*} = [\phi]_{E^*}^\top \widetilde{\BB}^\top [\psi]_{F^*}, \text{ where $\widetilde{\BB} := \Big(\widetilde{B}(\psi^{\ff_i}, \phi^{\ee_j})\Big)$}.
    \end{align*}
    
    for all $\vv \in V$, $\ww \in W$, $\phi \in V^*$, and $\psi \in W^*$.
\end{lemma}

\begin{proof}
    %We first show the first equation of the first line. We have $B(\vv, \ww) = \vv^{\flat_1}(\ww)$, and the $1 \times m$ matrix $\vv^{\flat_1}(F)$ of $\vv^{\flat_1}:W \rightarrow K$ relative to $F$ is $\vv^{\flat_1}(F) = [\vv^{\flat_1}]_{F^*}^\top$ (recall Theorem \ref{ch::bilinear_forms_metric_tensors::thm::matrix_of_dual_vector_as_trtransposed_coords}), so applying the characterizing property of [...] matrices (see Derivation [...]) gives $B(\vv, \ww) = \vv^{\flat_1}(\ww) = \vv^{\flat_1}(F) [\ww]_F = [\vv^{\flat_1}]_{F^*}^\top [\ww]_F$. We know $[\vv^{\flat_1}]_{F^*} = \BB [\vv]_E$ from the fourth equation of Theorem \ref{ch::bilinear_forms_metric_tensors::thm::vectors_dual_vectors_metric_tensor}, and so obtain $B(\vv, \ww) = (\BB [\vv]_E)^\top [\ww]_F = [\vv]_E^\top \BB [\ww]_F$, as desired.
  
    To show the first equation of the first line, we follow a derivation similar in spirit to that of a matrix relative to bases of a linear function between vector spaces. 
    
    We first consider the special case in which the vector spaces are $V = K^n$ and $W = K^m$ and where the bases are the standard bases for $K^n$ and $K^n$, $E = \sE$ and $F = \sF$. In this special case, we have $B(\vv, \ww) = \sum_{ij} ([\vv]_\sE)^i ([\ww]_\sE)^j B_{ij} = \sum_i ([\vv]_\sE)^i \sum_j B_{ij} ([\ww]_\sE)^j = \sum_i ([\vv]_\sE)^i ([\BB \ww]_\sF)^i = [\vv]_\sE \cdot [\BB \ww]_\sF = [\vv]_\sE^\top \BB [\ww]_\sF$, so the special case is proven. 
    
    For the case when $V$ and $W$ are $n$- and $m$- dimensional vector spaces with bases $E$ and $F$, consider the bilinear form $C$ on $K^n$ and $K^m$ defined by $C(\vv, \ww) := B([\vv]_E, [\ww]_F)$. By the special case, we have $B(\vv, \ww) = [\vv]_E^\top \CC [\ww]_F$, where $\CC = (C(\ee_i, \ff_j))$. We have $\CC = (C(\ee_i, \ff_j)) = (B([\ee_i]_E, [\ff_i]_F)) = (B(\see_i, \sff_i)) = \BB$, so $B(\vv, \ww) = [\vv]_E^\top \BB [\ww]_F$, as claimed.
    
    The first equation of the second line is obtained by applying the first line to the induced nondegenerate bilinear form $\widetilde{B}$ on $W^*$ and $V^*$, and the second equation on each line follows by transposing the first equation on each line.
\end{proof}
 
\begin{theorem}
\label{ch::bilinear_forms_metric_tensors::thm::matrices_musical_isos}
    (Matrices of the musical isomorphisms).
    
    Let $V$ and $W$ be finite-dimensional vector spaces with bases $E$ and $F$, let $B$ be a bilinear form on $V$ and $W$, and let $E^*$ and $F^*$ be the bases for $V^*$ and $W^*$ induced by $E$ and $F$. We have

    \begin{align*}
        [\flat_1(E)]_{F^*} = \BB^\top \text{ and } [\flat_2(F)]_{E^*} = \BB, \text{ where $\BB := \Big(B(\ee_i, \ee_j)\Big)$} \\
        [\widetilde{\flat_1}(F^*)]_{E^{**}} = (\widetilde{\BB})^\top \text{ and } [\flat_2(E^*)]_{F^{**}} = \widetilde{\BB}, \text{ where $\widetilde{\BB} := \Big(\widetilde{B}(\ee_i, \ee_j)\Big)$}. 
    \end{align*}
\end{theorem}

\begin{proof}
    The first line is true because we have $([\flat_1(E)]_{F^*})^i{}_j = ([\flat_1(\ee_j)]_{F^*})^i = ([\ee_j^{\flat_1}]_{F^*})^i = \ee_j^{\flat_1}(\ff_i) = B(\ee_j, \ff_i)$ and \\ $([\flat_2(F)]_{E^*})^i{}_j = ([\flat_2(\ff_j)]_{E^*})^i = ([\ff_j^{\flat_2}]_{E^*})^i = \ff_j^{\flat_2}(\ee_i) = B(\ee_i, \ff_j)$. The second line follows by applying the first line to the bilinear form $\widetilde{B}:W^* \times V^* \rightarrow K$.
\end{proof}
 
\begin{remark}
    (Indexing conventions for entries of matrices of bilinear forms).
    
    Consider the hypotheses of the previous theorem and additionally assume that $V = W$. Then ${B \in \LLLL(V \times V \rightarrow K)}$, so $B$ can be identified with a $(0, 2)$ tensor on $V$, and $\widetilde{B} \in \LLLL(V^* \times V^* \rightarrow K)$, so $\widetilde{B}$ can be identified with a $(2, 0)$ tensor on $V$. (Recall Remark \ref{ch::bilinear_forms_metric_tensors::rmk::linear_k_forms_0_k_tensors}). To make sure that we are following the convention regarding coordinates of tensors established in Definition \ref{ch::motivated_intro::defn::pq_tensor_coords}, we use $B_{ij}$ to denote the $ij$ entry of the matrix $\BB$ and use $B^{ij}$ to denote the $ij$ entry of the matrix $\widetilde{\BB}$.
\end{remark}
 
\begin{theorem}
    \label{ch::bilinear_forms_metric_tensors::thm::B_Btilde_kind_of_inverses}
    ($\flat_1$ and $\widetilde{\flat}_1$ are ``kind of'' inverses).
    
    Let $V$ and $W$ be finite-dimensional vector spaces with bases $E$ and $F$, let $E^* = \{\phi^{\ee_1}, ..., \phi^{\ee_n}\}$ and $F^* = \{\psi^{\ff_1}, ..., \psi^{\ff_m}\}$ be the bases for $V^*$ and $W^*$ induced by $E$ and $F$, and let $E^{**}$ and $F^{**}$ be the bases for $V^{**}$ and $W^{**}$ induced by $E^*$ and $F^*$. Lastly, let $B$ be a nondegenerate bilinear form on $V$ and $W$ with induced musical isosmorphisms $\flat_1, \flat_2$, and let $\widetilde{B}$ be the induced nondegenerate bilinear form on $W^*$ and $V^*$ with induced musical isomorphisms $\widetilde{\flat}_1, \widetilde{\flat}_2$.
    
    The musical isomorphisms $\flat_1$ and $\widetilde{\flat}_1$ are ``kind of'' inverses in the following sense:
     
    \begin{enumerate}
         \item $\widetilde{\flat}_1 \circ \flat_1 = (\vv \mapsto \Phi_\vv)$
        \item The matrices of $\widetilde{\flat}_1$ and $\flat_1$ relative to the appropriate bases are inverses: $[\widetilde{\flat}_1(F^*)]_{E^{**}} [\flat_1(E)]_{F^*} = \II$.
        \item $\BB^{-1} = \widetilde{\BB}$, where $\widetilde{\BB} = \Big(\widetilde{B}(\psi^{\ff_i}, \phi^{\ee_j})\Big)$.
    \end{enumerate}
\end{theorem}
 
\begin{proof}
    \hspace{0mm} \\
    \begin{enumerate}
        \item We have: $\widetilde{\flat}_1 \circ \flat_1 = (\vv \mapsto \Phi_\vv) \iff  ((\vv^{\flat_1})^{\widetilde{\flat}_1} = \Phi_\vv \text{ for all $\vv \in V$}) \iff
        ((\vv^{\flat_1})^{\widetilde{\flat}_1}(\phi) = \Phi_\vv(\phi) \text{ for all $\vv \in V, \phi \in V^*$})$. Since $(\vv^{\flat_1})^{\widetilde{\flat}_1}(\phi) = \widetilde{B}(\vv^{\flat_1}, \phi) = B(\vv, \phi^{\sharp_2}) = \vv^{\flat_1}(\phi^{\sharp_2})$ and $\Phi_\vv(\phi) = \phi(\vv)$, an equivalent condition is $(\vv^{\flat_1}(\phi^{\sharp_2}) = \phi(\vv) \text{ for all $\vv, \ww \in V$})$. Then since $\sharp_2$ is an isomorphism we can substitute $\ww \in W$ for $\phi^{\sharp_2}$ to obtain another equivalent condition, $(\vv^{\flat_1}(\ww) = \ww^{\flat_2}(\vv) \text{ for all $\vv \in V, \phi \in V^*$})$, which is equivalent to the obviously true condition $(B(\vv, \ww) = B(\vv, \ww) \text{ for all $\vv, \ww \in V$})$.
        
        \item The matrix of a composition of a linear function is equal to the matrix-matrix product of the matrices (with respect to the appropriate bases) of the functions in the composition, so we have $[\widetilde{\flat}_1 \circ \flat_1]_{E^{**}} = [\widetilde{\flat}_1(F^*)]_{E^{**}} [\flat_1(E)]_{F^*}$. Since $\widetilde{\flat}_1 \circ \flat_1 = (\vv \mapsto \Phi_\vv)$, we can prove the desired fact- that the matrices on the right side of this last equation are inverses- by showing that the matrix of $(\vv \mapsto \Phi_\vv) = \FF$ relative to $E$ and $E^{**}$ is $\II$. To do so, define $\FF(\vv) = \Phi_\vv$, so that the matrix of $(\vv \mapsto \Phi_\vv) = \FF$ relative to $E$ and $E^{**}$ is $[\FF(E)]_{E^{**}}$. The $ij$ entry of this matrix is $([\FF(\ee_j)]_{E^{**}})^i = ([\Phi_{\ee_j}]_{E^{**}})^i = \Phi_{\ee_j}(\phi^{\ee_i})$, where $\phi^{\ee_i}$ denotes the $i$th basis vector in $E^*$. Since $\Phi_{\ee_j}(\phi^{\ee_i}) = \phi^{\ee_i}(\ee_j) = \delta^i{}_j$, the matrix itself is $\II$.
        \item Theorem \ref{ch::bilinear_forms_metric_tensors::thm::matrices_musical_isos} tells us that $[\flat_1(E)]_{F^*} = \BB^\top$ and $[\widetilde{\flat}_1(F^*)]_{E^{**}} = (\widetilde{\BB})^\top$. Using these results with (2), we obtain $(\widetilde{\BB})^\top \BB^\top = \II$. Take the transpose of both sides to obtain $\widetilde{\BB} \BB = \II$, as desired.
    \end{enumerate}
\end{proof}

\subsection*{Metric tensors}

\begin{defn}
    (Metric tensor).
    
    Let $V$ be a finite-dimensional vector space. A nondegenerate bilinear form $g$ on $V$ that is also \textit{symmetric}, in the sense that $g(\vv, \ww) = g(\ww, \vv)$ for all $\vv, \ww \in V$, is called a \textit{metric tensor on $V$}.
\end{defn}

\begin{remark}
    (Inner products are metric tensors).
    
    Let $V$ be a vector space over a field $K$. Recall from Definition     \ref{ch::lin_alg::defn::inner_product} that an \textit{inner product} on $V$ is a nondegenerate symmetric bilinear form on $V$ and $V$.
\end{remark}

\begin{defn}
    (The notation $\flat$ and $\sharp$).
    
    When $V$ is a finite-dimensional vector space and there is a metric tensor $g$ on $V$, then the musical isomorphisms ${\flat_1:V \rightarrow V^*}$ and ${\flat_2:V \rightarrow V^*}$ induced by $g$ are the same because $g$ is symmetric. In this scenario, we define $\flat := \flat_1 = \flat_2$ and $\sharp := \sharp_1 = \flat_1^{-1} = \sharp_2 = \flat_2^{-1}$.
\end{defn}

\subsubsection*{Misc.}

\begin{defn}
    (Reciprocal bases).

    Bases $E = \{\ee_1, ..., \ee_n\}$ and $F = \{\ff_1, ..., \ff_n\}$ for a vector space with a nondegenerate bilinear form $B$ are said to be \textit{reciprocal} iff $B(\ee_i, \ff_j) = \delta^i{}_j$.

    Recall that if $B$ is an inner product, then a basis $E = \{\ee_1, ..., \ee_n\}$ is said to be \textit{orthonormal} iff $B(\ee_i, \ee_j) = \delta^i{}_j$. In other words, $E$ is orthonormal iff it is reciprocal with itself.
\end{defn}

\begin{theorem}
    Two bases of a vector space with a nondegenerate bilinear form are reciprocal iff $\flat_1$ and $\flat_2$ send basis vectors of one basis to dual basis vectors of the other.
\end{theorem}

\begin{proof}
    We have: $B(\ee_i, \ff_j) = \delta^i{}_j \iff B(\ee_i, \ff_j) = \psi^{\ff_i}(\ff_j) \iff \ee_i^{\flat_1}(\ff_j) = \psi^{\ff_i}(\ff_j) \iff \ee_i^{\flat_1}(\vv) = \psi^{\ff_i}(\vv) \text{ for all $\vv \in V$} \iff \ee_i^{\flat_1} = \psi^{\ff_i}$. Overall, we have $B(\ee_i, \ff_j) = \delta^i{}_j \iff \ee_i^{\flat_1} = \psi^{\ff_i}$. A similar argument shows $B(\ee_i, \ff_j) = \delta^i{}_j \iff \ff_i^{\flat_2} = \phi^{\ee_i}$.
\end{proof}

\begin{remark}
\label{ch::bilinear_forms_metric_tensors::thm::musical_iso_unique_self_dual_iso}
    (Naturality of self-duality).

    The above theorem shows that the ``unnatural'' isomorphism $V \rightarrow V^*$ sending $\ee_i \mapsto \phi^{\ee_i}$ that was discussed earlier in Remark \ref{ch::motivated_intro::rmk::unnatural_iso_V_V*} actually becomes natural exactly whenever we have a nondegenerate bilinear form $B$ on $V$ with $B(\ee_i, \ee_j) = \delta^i{}_j$.
\end{remark}

\newpage


\newpage

\section*{Coordinates with a nondegenerate bilinear form}

\begin{theorem}
\label{ch::bilinear_forms_metric_tensors::thm::vectors_dual_vectors_metric_tensor}

    (Relationship between coordinates of vectors and dual vectors for vector spaces).
    
    Let $V$ and $W$ be finite-dimensional vector spaces over a field $K$ with bases $E$ and $F$, let $E^* = \{\phi^{\ee_1}, ..., \phi^{\ee_n}\}$ and $F^* = \{\psi^{\ff_1}, ..., \psi^{\ff_n}\}$ be the bases for $V^*$ and $W^*$ induced by $E$ and $F$, and let $E^{**}$ and $F^{**}$ be the bases for $V^{**}$ and $W^{**}$ induced by $E^*$ and $F^*$. Lastly, let $B$ be a nondegenerate bilinear form on $V$ and $W$ with induced musical isosmorphisms $\flat_1, \flat_2$, and let $\widetilde{B}$ be the induced nondegenerate bilinear form on $W^*$ and $V^*$ with induced musical isomorphisms $\widetilde{\flat}_1, \widetilde{\flat}_2$. 

    Then we have
    
    \begin{align*}
        [\vv]_E = \BB^{-\top} [\vv^{\flat_1}]_{F^*} &\text{ and }
        [\ww]_F = \BB^{-1} [\ww^{\flat_2}]_{E^*} \\
        [\phi]_{E^*} = \BB [\phi^{\sharp_2}]_F &\text{ and }
        [\psi]_{F^*} = \BB^\top [\psi^{\sharp_1}]_E,
    \end{align*}
    
    for all $\vv \in V$, $\ww \in W$, $\phi \in V^*$, and $\psi \in W^*$, where $\BB = (B(\ee_i, \ee_j))$ and $\widetilde{\BB} = (\widetilde{B}(\psi^{\ff_i}, \phi^{\ee_j}))$.

    \vspace{.25cm}

    In the special case when $V = W$ and $E = F$, we have

    \begin{align*}
        [\vv]_E &= \BB^{-\top} [\vv^{\flat_1}]_{E^*} = \BB^{-1} [\vv^{\flat_2}]_{E^*} \\
        [\phi]_{E^*} &= \BB^\top [\phi^{\sharp_1}]_E = \BB [\phi^{\sharp_2}]_E,
    \end{align*}

    for all $\vv \in V, \phi \in V^*$, where $\BB = (B(\ee_i, \ee_j))$ and $\widetilde{\BB} = (\widetilde{B}(\phi^{\ee_i}, \phi^{\ee_j}))$.
\end{theorem}

\begin{proof}
    We will prove (1) that $[\psi]_{F^*} = \BB^\top [\psi^{\sharp_1}]_E$ for all $\psi \in W^*$ and (2) that $[\vv]_E = \BB^{-\top} [\vv^{\flat_1}]_{F^*}$ for all $\vv \in V$. The top right equation is obtained by applying (1) to the nondegenerate bilinear form $C$ on $W$ and $V$ defined by $C(\ww, \vv) := B(\vv, \ww)$, and the bottom left equation is similarly obtained by applying (2) to the nondegenerate bilinear form $\widetilde{C}$ on $V^*$ and $W^*$ defined by $\widetilde{C}(\phi, \psi) := \widetilde{B}(\psi, \phi)$.

    First, we prove (1). Recall from Definition [...] that the matrix $[\flat_1(E)]_{F^*}$ of $\flat_1:V \rightarrow W^*$ relative to $E$ and $F^*$ satisfies the characterizing property $[\flat_1(\vv)]_{F^*} = [\flat_1(E)]_{F^*} [\vv]_E$ for all $\vv \in V$. That is, $[\vv^{\flat_1}]_{F^*} = [\flat_1(E)]_{F^*} [\vv]_E$ for all $\vv \in V$. We know from Theorem \ref{ch::bilinear_forms_metric_tensors::thm::matrices_musical_isos} that $[\flat_1(E)]_{F^*} = \BB^\top$, so we have $[\vv^{\flat_1}]_{F^*} = \BB^\top [\vv]_E$. Since $\flat_1$ is an isomorphism, we can replace $\vv^{\flat_1} \in W^*$ with an arbitrary $\psi \in W^*$ to obtain $([\psi]_{F^*} = \BB^\top [\psi^{\sharp_1}]_E \text{ for all } \psi \in W^*)$, as desired.
    
    Now we prove (2). Applying (1) to the induced nondegenerate bilinear form $\widetilde{B}:W^* \times V^* \rightarrow K$, we obtain an analogous statement to (1) in which $\psi \in W^*$ is replaced with $\Phi \in V^{**}$, $F^*$ is replaced with $E^{**}$, $E$ is replaced with $F^*$, $\sharp_1:W^* \rightarrow V$ is replaced with $\widetilde{\sharp}_1:V^{**} \rightarrow W^*$, and $\BB$ is replaced with $\widetilde{\BB}$: we have $[\Phi]_{E^{**}} = \widetilde{\BB}^\top [\Phi^{\widetilde{\sharp}_1}]_{F^*}$ for all $\Phi \in V^{**}$. Substitute $\Phi_\vv$ in for $\Phi$ and use the fact $([\Phi_\vv]_{E^{**}})^i = ([\vv]_E)^i$ from Theorem \ref{ch::bilinear_forms_metric_tensors::thm::coords_vector_dual_vector} to obtain the statement ``$[\vv]_E = \widetilde{\BB}^\top [\Phi_\vv^{\widetilde{\sharp}_1}]_{F^*}$ for all $\vv \in V$''. Notice that if we show that $\Phi_\vv^{\widetilde{\sharp}_1} = \vv^{\flat_1}$ for all $\vv \in V$, then this equation involves only $\vv$ and not $\Phi_\vv$, and becomes $[\vv]_E = \widetilde{\BB}^\top [\vv^{\flat_1}]_{F^*}$. This condition does hold: we have $(\Phi_\vv^{\widetilde{\sharp}_1} = \vv^{\flat_1} \text{ for all } \vv \in V) \iff (\Phi_\vv = (\vv^{\flat_1})^{\widetilde{\flat}_1} \text{ for all } \vv \in V) \iff ((\vv \mapsto \Phi_\vv) = \widetilde{\flat}_1 \circ \flat_1)$, where this last condition is just the first item of Theorem \ref{ch::bilinear_forms_metric_tensors::thm::B_Btilde_kind_of_inverses}. So we know $[\vv]_E = \widetilde{\BB}^\top [\vv^{\flat_1}]_{F^*}$ for all $\vv \in V$. Use the fact that $\widetilde{\BB} = \BB^{-1}$ from Theorem \ref{ch::bilinear_forms_metric_tensors::thm::B_Btilde_kind_of_inverses} to obtain $([\vv]_E = (\BB^{-1})^\top [\vv^{\flat_1}]_{F^*} = (\BB^{-\top}) [\vv^{\flat_1}]_{F^*} \text{ for all $\vv \in V$}$, as desired.
\end{proof}

\begin{remark}
    (Metric tensors in physics).
    
    In physics, the typical situation is to have a vector space with metric tensor $g$. (Recall that a metric tensor on a vector space $V$ is a nondegenerate symmetric bilinear form on $V$ and $V$). In this situation, the symmetry of $g$ further simplifies the special case of the above theorem:
    
    \begin{align*}
        [\vv]_E = \gg^{-1} [\vv^\flat]_{E^*} &\iff [\vv^\flat]_{E^*} = \gg [\vv]_E \\
        [\phi]_{E^*} = \gg [\phi^\sharp]_E &\iff [\phi^\sharp]_E = \gg^{-1} [\phi]_{E^*}.
    \end{align*}
    
    Physicists also make the definitions $v^i := ([\vv]_E)^i$, $v_i := ([\vv^\flat]_{E^*})_i$, $\phi_i := ([\phi]_{E^*})_i$, and $\phi^i := ([\phi^\sharp]_E)^i$. With these definitions, the above equations become
    
    \begin{align*}
        v^i = \sum_j g^{ij} v_j &\iff v_i = \sum_j g_{ij} v^j \\
        \phi_i = \sum_j g_{ij} \phi^j &\iff \phi^i = \sum_j g^{ij} \phi_j.
    \end{align*}
    
    This notation has the advantage of being compact, and it's what I would personally use when doing calculations. However, it is best for reference materials such as this one to introduce the relations between $v^i$ and $v_i$ with notation that does involve explicit mention of the basis $E$.
\end{remark}

\begin{deriv}
    (Using a metric tensor to convert between vectors and dual vectors in a $(p, q)$ tensor).
    
    Let $V$ be a finite-dimensional vector space with basis $E = \{\ee_1, ..., \ee_n\}$, let $E^* = \{\phi^{\ee_1}, ..., \phi^{\ee_n}\}$ be the dual basis for $V^*$ induced by $E$, and consider a $(p, q)$ tensor $\TT \in T_{p,q}(V)$ with coordinates $T^{i_1 ... i_p}{}_{j_1 ... j_q}$ relative to $E$ and $E^*$,

    \begin{align*}
        \TT = \sum_{\substack{i_1, ..., i_p \in \{1, ..., n\} \\ j_1, ..., j_q \in \{1, ..., m\}}} T^{i_1 ... i_p}{}_{j_1 ... j_q} \ee_{i_1} \otimes ... \otimes \ee_{i_k} \otimes ... \otimes \ee_{i_p} \otimes \phi^{\ee_{j_1}} \otimes ... \otimes \phi^{\ee_{j_q}}.
    \end{align*}

    If we have a metric tensor $g$ on $V$, then we can send a basis $(p, q)$ tensor $\SS^{i_1 ... i_p}{}_{j_1 ... j_q}$ to a $(p - 1, q + 1)$ tensor by applying the map $\flat:V \rightarrow V^*$ to one of the $p$ vectors (as opposed to one of the $q$ dual vectors): 
    
    \begin{align*}
        \SS^{i_1 ... i_p}{}_{j_1 ... j_q} := \ee_{i_1} \otimes ... \otimes &\ee_{i_k} \otimes ... \otimes \ee_{i_p} \otimes \phi^{\ee_{j_1}} \otimes ... \otimes \phi^{\ee_{j_q}} \\
        &\qquad \longmapsto \\
        \ee_{i_1} \otimes ... \otimes &\ee_{i_k}^\flat \otimes ... \otimes \ee_{i_p} \otimes \phi^{\ee_{j_1}} \otimes ... \otimes \phi^{\ee_{j_q}}.
    \end{align*}
    
    Using the equation $[\phi]_{E^*} = \gg [\phi^\sharp]_E \iff ([\phi]_{E^*})^r = \sum_{j = 1}^n g_{rj} ([\phi^\sharp]_E)^j$ from the previous remark, we compute $\ee_{i_k}^\flat$ to be 
    
    \begin{align*}
        \ee_{i_k}^\flat = \sum_{r = 1}^n ([\ee_{i_k}^\flat]_{E^*})^r \phi^{\ee_r} =
        \sum_{r = 1}^n \Big( \sum_{j = 1}^n g_{rj} ([\ee_{i_k}]_E)^j \Big) \phi^{\ee_r} = \sum_{r = 1}^n \Big( \sum_{j = 1}^n g_{rj} \delta^j{}_{i_k} \Big) \phi^{\ee_r}
        = \sum_{r = 1}^n g_{r i_k} \phi^{\ee_r},
    \end{align*}
    
    so $\SS^{i_1 ... i_p}{}_{j_1 ... j_q}$ is sent to
    
    \begin{align*}
        &\ee_{i_1} \otimes ... \otimes \ee_{i_{k - 1}} \otimes \sum_{r = 1}^n \Big( g_{r i_k} \phi^{\ee_r} \Big) \otimes \ee_{i_{k + 1}} \otimes ... \otimes \ee_{i_p} \otimes \phi^{\ee_{j_1}} \otimes ... \otimes \phi^{\ee_{j_q}} \\
        &= \sum_{r = 1}^n g_{r i_k} \Big( \ee_{i_1} \otimes ... \otimes \ee_{i_{k - 1}} \otimes \phi^{\ee_r} \otimes \ee_{i_{k + 1}} \otimes ... \otimes \ee_{i_p} \otimes \phi^{\ee_{j_1}} \otimes ... \otimes \phi^{\ee_{j_q}} \Big).
    \end{align*}
    
    Thus, if the coordinates of $\TT$ relative to $E$ and $E^*$ were originally $T^{i_1 ... i_p}{}_{j_1 ... j_q}$, then they get sent to 
    
    \begin{align*}
        \sum_{r = 1}^n g_{r i_k} T^{i_1 ... i_{k - 1}}{}_{r}{}^{i_{k + 1} ... i_p}{}_{j_1 ... j_q}.
    \end{align*}
    
    Following a similar process to above, we can use the other musical isomorphism, the sharp map $\sharp = \flat^{-1}$, to convert a $(p, q)$ tensor to a $(p + 1, q - 1)$ tensor. This approach would send $T^{i_1 ... i_p}{}_{j_1 ... j_q}$ to
    
    \begin{align*}
        \sum_{r = 1}^n g^{r j_k} T^{i_1 ... i_p}{}_{j_1 ... j_{k - 1}}{}^{r}{}_{j_{k + 1} ... j_q}.
    \end{align*}
    
    So, using the notation at the end of the previous remark, we have the following ``index lowering'' and ``index raising'' mappings:
    
    \begin{empheq}[box = \fbox]{align*}
        T^{i_1 ... i_p}{}_{j_1 ... j_q} &\mapsto \sum_{r = 1}^n g_{r i_k} T^{i_1 ... i_{k - 1}}{}_{r}{}^{i_{k + 1} ... i_p}{}_{j_1 ... j_q} \quad \text{(index lowering)} \\
        T^{i_1 ... i_p}{}_{j_1 ... j_q} &\mapsto \sum_{r = 1}^n g^{r j_k} T^{i_1 ... i_p}{}_{j_1 ... j_{k - 1}}{}^{r}{}_{j_{k + 1} ... j_q} \quad \text{(index raising)}
    \end{empheq}
\end{deriv}

\begin{remark}
    (Index raising and lowering in greater generality).

    It is possible to prove a version of the above result that applies when we have a tensor from $V_1 \otimes ... \otimes V_k$, where ${V_i \in \{V, V^*, W, W^*\}}$, and a not-necessarily-symmetric nondegenerate bilinear form on finite-dimensional vector spaces $V$ and $W$.
    
    In this situation, we wouldn't necessarily have $\flat_1 = \flat_2$, $\sharp_1 = \sharp_2$, and $\BB = \BB^\top$. The full generality of the previous theorem would imply that there would be two different index-lowering operations (one involving multiplication by $B_{i_k r}$ and one involving multiplication by $B_{r i_k}$) and two different index-raising operations (one involving multiplication by $B_{j_k r}$ and one involving multiplication by $B^{r j_k}$).
    
    If the form of $V_1 \otimes ... \otimes V_k$ were set in stone for our purposes, e.g., $V_1 \otimes ... \otimes V_k = V \otimes W^* \otimes V^* \otimes W^{\otimes 2}$, then we would have a natural way to distinguish between indices that are lowered or raised as a result of applying $\flat_1$ or $\sharp_1$ and indices that are lowered or raised as a result of applying $\flat_2$ or $\sharp_2$: if an index corresponds to $V$ or $W^*$, then $\flat_1$ and $\sharp_1$ are applicable, and if an index corresponds to $W$ or $V^*$, then $\flat_2$ and $\sharp_2$ are applicable.

    If the form of $V_1 \otimes ... \otimes V_k$ weren't set in stone, however- perhaps because we wanted to investigate with as much generality as possible- then things would be much uglier. We would have to distinguish between index operations from resulting from $\flat_1$ and $\sharp_1$ and index operations resulting from $\flat_2$ and $\sharp_2$.    This could be achieved by putting primes $'$ or tildes $\sim$ on indices that are raised or lowered with $\flat_2$ or $\sharp_2$, and leaving indices that are raised or lowered with $\flat_1$ or $\sharp_1$ alone. But as you can imagine, that is quite cumbersome.

    % Here is a derivation of the above remark. Consider the equations of the previous theorem,

    % \begin{align*}
    %     [\vv]_E = \BB^{-\top} [\vv^{\flat_1}]_{F^*} &\text{ and }
    %     [\ww]_F = \BB^{-1} [\ww^{\flat_2}]_{E^*} \\
    %     [\phi]_{E^*} = \BB [\phi^{\sharp_2}]_F &\text{ and }
    %     [\psi]_{F^*} = \BB^\top [\psi^{\sharp_1}]_E.
    % \end{align*}

    % \begin{itemize}
    %     \item applying $\flat_1$ necessitates using expression for $[\psi]_{F^*}$, which is found in the bottom right equation of the previous theorem. since this equation involves matrix multiplication by $\BB^\top$, multiplication by $B_{r i_k}{}^\top = B_{i_k r}$ appears in the end result. this would be an index-lowering operation.

    %     \item applying $\sharp_1$ necessitates using expression for $[\vv]_E$, which is found in the top left equation of the previous theorem. since this equation involves matrix multiplication by $\BB^{-\top}$, multiplication by $(B^{r j_k})^\top = B^{j_k r}$ appears in the end result. this would be an index-raising operation.
    % \end{itemize}

    % and

    % \begin{itemize}
    %     \item applying $\flat_2$ necessitates using expression for $[\phi]_{E^*}$, which is found in the bottom left equation of the previous theorem. since this equation involves matrix multiplication by $\BB$, multiplication by $B_{r i_k}$ appears in the end result. this would be an index-lowering operation.

    %     \item applying $\sharp_2$ necessitates using expression for $[\ww]_F$, which is found in the top right equation of the previous theorem. since this equation involves matrix multiplication by $\BB^{-\top}$, multiplication by $B^{r j_k}$ appears in the end result. this would be an index-raising operation.
    % \end{itemize}
\end{remark}