\chapter*{For the future}

\begin{itemize}
    \item vector fields should be denoted as $v$, not $\VV$, since a continuous map sending $\pp \mapsto v_\pp$ is a vector field
    \item in ``coordinates of tensors'' section of ``bilinear forms'' chapter, make sure never see expressions like $([\FF]_E)^i_j$, only $([\FF]_E)^i{}_j$
\end{itemize}

\vspace{.5cm}

This is a list of items that must be completed before I consider this book truly complete.

\begin{itemize}
    \item Finish cross product section
    \item When Hodge-dual is used to generalize vector calculus theorems, remind reader that Hodge-dual was introduced in cross product section. 
    
    Emphasize that Hodge-dual is basically a generalized cross product. This perspective implies the abstract definition of the Hodge-dual that involves orthogonal subspaces, which is often pulled out of thin air in other presentations. (So maybe prove that the Hodge-dual satisfies this condition involving orthogonal subspaces as a result of it satisfying cross-product like condition. Now that I think about it, this proof will be analogous to proving that $\vv \times \ww$ is perpendicular to $\vv$ and $\ww$ by using the defn of $\times$. This means that definition of Hodge dual will be in cross product section, so can remove definition near vector calculus stuff).
    \item Add more explanatory text in-between definitions, lemmas, theorems where needed.
    \item Figure out the unexplained step in Theorem \ref{ch::diff_forms::thm::integral_of_diff_form_actual_function_single_chart}. Somehow, $d\xx \Big( \frac{\pd}{\pd \tx^i_{(V, \yy)}} \Big) = \frac{\pd \xx}{\pd \tx^i_{(V, \yy)}}$. 
    \item Finish the section in ``Manifolds'' on frames and coframes.
    \item Figure out why the ``mnemonic'' of the ``Manifolds'' chapter works.
    \item Add the divergence theorem and the less general Stokes' theorem (and Green's theorem) after the generalized Stokes' theorem.
    \item Potentially relocate the section about differential forms in $\tOmega^k(M)$ that act on tangent vectors. Potentially the presentation of  tensors/differential forms as actual (pointwise) multilinear functions.
    \item Explain more explicitly how the cancellation in part 3 of the proof of \ref{ch::diff_forms::theorem::stokes_on_a_smooth_chart} occurs. (``all the internal boundaries in the sum $\sum_{C \in D_N(\cl(\H^k))} \int_{\pd C}$ cancel, since each boundary appears twice with opposite orientations'').
\end{itemize}