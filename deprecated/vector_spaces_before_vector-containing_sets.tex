\subsection*{Properties of $\R^n$}

There is one last fundamental property of $\R^n$ to cover: $\R^n$ is ``spanned'' by a certain set of vectors. Of course, before we can present this property, we need to know what it means for a set to be ``spanned'' by vectors.

\begin{defn}
    (Span).
    
    Given a subset $W \subseteq \R^n$, we define the \textit{span} of $W$ to be the set of all convergent weighted sums involving vectors from $W$:
    
    \begin{align*}
        \spann(W) := \Big\{ \sum_{c \in X, \vv \in U} c \vv \mid X \subseteq \R \text{ and } U \subseteq W \Big\}.
    \end{align*}
    
    In particular, when $W = \{\vv_1, ..., \vv_k\}$ is a finite set, we have
    
    \begin{align*}
        \spann(W) = \spann(\{\vv_1, ..., \vv_k\}) = \Big\{ c_1 \vv_1 + ... + c_k \vv_k \mid c_1, ..., c_k \in \R \Big\}.
    \end{align*}
    
    Geometrically, $\spann(\vv_1, ..., \vv_k)$ is the $k$-dimensional plane spanned by $\vv_1, ..., \vv_k$ that is embedded in $\R^n$. For example, if $\vv, \ww \in \R^3$, then $\spann(\vv, \ww) = \{ c \vv + d \ww \mid c, d \in \R \}$ is the $2$-dimensional plane spanned by $\vv$ and $\ww$ in $\R^3$.
\end{defn}

\begin{deriv}
    ($\R^n$ is spanned by $n$ vectors from $\R^n$).
    
    For any vector $\vv = \begin{pmatrix} v_1 \\ \vdots \\ v_n \end{pmatrix} \in \R^n$, we have
    
    \begin{align*}
        \vv
        =
        \begin{pmatrix} v_1 \\ v_2 \\ \vdots \\ v_n \end{pmatrix}
        =
        v_1 \begin{pmatrix} 1 \\ 0 \\ \vdots \\ 0 \end{pmatrix}
        + ... +
        v_n \begin{pmatrix} 0 \\ \vdots \\ 0 \\ 1 \end{pmatrix}.
    \end{align*}
    
    We use special notation for the vectors in the above sum, and define $\see_i$ to be the vector that has an $i$th component of $1$, with all other components being $0$. Thus, the above can be written as
    
    \begin{align*}
        \vv = v_1 \see_1 + ... + v_n \see_n.
    \end{align*}
    
    Notice that the above implies $\vv \in \spann(\see_1, ..., \see_n)$. Since $\vv$ ranges over $\R^n$, this shows that $\R^n \subseteq \spann(\see_1, ..., \see_n)$. We clearly also have $\spann(\see_1, ..., \see_n) \subseteq \R^n$, so we can conclude that $\R^n = \spann(\see_1, ..., \see_n)$:
    
    \begin{align*}
        \R^n = \spann\Bigg( \underbrace{\begin{pmatrix} 1 \\ 0 \\ \vdots \\ 0 \end{pmatrix}}_{\see_1}, ..., \underbrace{\begin{pmatrix} 0 \\ 0 \\ \vdots \\ 1 \end{pmatrix}}_{\see_n} \Bigg).
    \end{align*}
\end{deriv}

\subsection*{Vector spaces over $\R$}

Before making our definition, we quickly generalize the notion of span so that is applicable in a more abstract setting.

\begin{defn}
    (Span).
    
    Let $W$ be a set for which there exist functions $+:W \times W \rightarrow W$ and $\cdot:\R \times W \rightarrow W$. As we did for $c \in \R$ and $\vv \in \R^n$, when we have $c \in \R$ and $\vv \in W$,  we write ``$c \vv$'' to mean ``$c \cdot \vv$''.
    
    We define the \textit{span} of $W$ to be the set of all convergent weighted sums involving elements of $W$, much as we did before:
    
    \begin{align*}
        \spann(W) := \Big\{ \sum_{c \in X, \vv \in U} c \vv \mid X \subseteq \R \text{ and } U \subseteq W \Big\}.
    \end{align*}
\end{defn}


\begin{remark}
    (Vector spaces over $\R$ are in some sense multidimensional planes).
    
    To gain an intuitive understanding for what a vector space over $\R$ is, focus on the above property (2), which specifies that a vector space over $\R$ is spanned by some set, rather than property (1). We ask: ``what is special about sets that are spanned?''
    
    To investigate this, consider  a set $V$ that is spanned by a set $W$ in the special case where $V$ is a subset of $\R^n$, for some positive integer $n$. It takes a little more machinery\footnote{We need to introduce the concept of \textit{dimension} and also know that the dimension of a vector space is unique. Then we can conclude there is a finite set that spans $V$.} to show so, but hopefully it is intuitive that there is a set of $n$ or fewer vectors that spans $V$. In other words, $V$ is a $k$-dimensional plane for some $k \leq n$.
    
    Thus we can think of vector spaces over $\R$ as being similar in some sense to multidimensional planes\footnote{This formalizes how ``$k$-dimensional'' vector spaces over $\R$ are similar to $k$-dimensional planes.
    
    Theorem \ref{ch::lin_alg::thm::same_dim_iff_isomorphic}, we will learn that every ``$k$-dimensional'' vector space over $\R$ is ``isomorphic'' to \textit{every} $k$-dimensional plane in $\R^n$.} in $\R^n$. Though, when we have a vector space over $\R$ that is spanned by infinitely many vectors, it is more accurate to think of that vector space as being similar to a plane with ``infinitely many dimensions''. 
\end{remark}
