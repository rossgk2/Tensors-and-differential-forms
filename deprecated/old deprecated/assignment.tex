Hi Ross,

Your 6 credit independent math comps topic is Computational Applications of Differential Forms:

Your independent comps project is about computational applications of differential forms. We appreciate that you have suggested a text, namely Crane's "Discrete Differential Geometry". This book, which is freely and legally available online, will be your main source. Of course, you may consult other texts, including the ones that you mentioned in your comps questionnaire.

Your main goal is to understand Crane's Chapter 6, which describes the core concept (Laplacian) and techniques of the theory. Also, you are expected to study one of the two applications --- Chapter 7 or Chapter 8 --- which depend on Chapter 6. To get to Chapter 6, you will need to digest the earlier chapters of the book. Chapters 1, 2, and 3 should be readable quickly, because they are intuitive and/or review for you. Chapters 4 and 5 will require detailed study, although parts of them are also review.

The products of your comps project are a talk and a paper. Both should be pitched at your fellow math majors. Your talk should convey the key concepts and motivations behind the definitions, theorems, proofs, and algorithms, without getting bogged down in long proofs or calculations. Your paper should present your project's material including the technical details. Both your talk and paper can benefit from concrete examples and illustrations, some of which might arise from Crane's exercises and code examples.

As alluded to above, you will give a 25 minute talk on this topic and complete a paper (typically 15-20 pages) summarizing what you learned. A detailed timetable is given on the department webpage (https://www.carleton.edu/math/major/comps/) and you should familiarize yourself with this. There will be a midpoint check-in due by the end of Week 4 (October 9) consisting of a 1-2 page outline of your paper and the first 4-5 slides of your presentation. You will get feedback on these slides from your peers. The public talk will be during Week 8 (November 3 or 5); you are expected to practice your talk with other majors and you will need a short abstract for the Goodsell Gazette prior to your talk. Because of the ongoing disruptions caused by COVID-19, the talk will be recorded and delivered online; we will confirm the details about the logistics later. Shortly after your talk, you will meet with two department faculty members to discuss your comps topic in more detail and expand on ideas that you didn’t have time to cover in your talk. The paper will be due Monday of Week 9 (November 9).

I will send you a reminder about the midpoint check-in and again to confirm the talk logistics. Note that this is intended to be an independent research and learning experience, but you are not restricted from discussing your topic with other students or faculty.

If you have any questions please contact me.

Good luck!

Andy