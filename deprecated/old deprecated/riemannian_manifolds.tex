\section{Riemannian manifolds}

\begin{defn}
    \smallcite{book::SM}{327} (Riemannian manifold).
    
    Let $M$ be a smooth manifold WWBOC. A \textit{Riemannian metric tensor} $g$ on $M$ is a smooth symmetric $\binom{0}{2}$ tensor field on $M$ that is positive-definite at each point. Thus, $g$ is an inner product on $T_\pp(M)$ for each $\pp \in M$.
    
    A \textit{Riemannian manifold WWBOC} is a tuple $(M, g)$, where $M$ is a smooth manifold WWBOC and $g$ is a Riemannian metric. An \textit{oriented Riemannian manifold WWBOC} is a Riemannian manifold WWBOC that has been given an orientation.
\end{defn}

\begin{theorem}
    \begin{itemize}
    \item \smallcite{book::SM}{329}``Every smooth manifold with or without boundary admits a Riemannian metric''. But there is no natural choice of metric
    \item \smallcite{book::SM}{332} flat metrics; not every metric is flat
\end{itemize}
\end{theorem}

\begin{defn}
    \smallcite{book::SM}{389} (Riemannian volume form).
    
    Let $(M, g)$ be a smooth oriented Riemannian manifold, let $\{\UU_1, ..., \UU_n\}$ be an ordered\footnote{We define an \textit{ordered frame}, analogous to an ordered basis, to be a frame in which the order of the vector fields matter.} orthonormal frame for $M$, and let $\{\UUU^1, ..., \UUU^n\}$ be the induced ordered orthonormal coframe. The \textit{Riemannian volume form $\omega_g$ corresponding to $\{\UU_1, ..., \UU_n\}$} is defined to be
    
    \begin{align*}
        \omega_g := \UUU^1 \wedge ... \wedge \UUU^n.
    \end{align*}
\end{defn}

\begin{theorem}
    \smallcite{book::SM}{389-390}(Riemannian volume form in local coordinates).
    
    Let $(M, g)$ be a smooth oriented Riemannian manifold, let $\{\UU_1, ..., \UU_n \}$ be an ordered orthonormal frame for $M$, and let $\{\UUU^1, ..., \UUU^n\}$ be the induced ordered orthonormal coframe. If $(U, \xx)$ is a smooth chart on $M$, then the Riemannian volume form is on $U$ expressed as
    
    \begin{align*}
        \omega_g = \det \Big( \frac{\pd \UUU_j}{\pd x^i} \Big) dx^1 \wedge ... \wedge dx^n.
    \end{align*}
\end{theorem}

\begin{proof}
    Using Theorem \ref{ch::manifolds::thm::change_coords_coord_frames}, which expresses the change of coordinates for frames in terms of coordinate frames, we have
    
    \begin{align*}
        \frac{\pd}{\pd x^i} = \sum_{j = 1}^n \frac{\pd \UUU_j}{\pd x^i} \frac{\pd}{\pd \UUU_j}.
    \end{align*}
    
    Since $\omega_g := \UUU^1 \wedge ... \wedge \UUU^n$ is a smooth differential $n$-form on $U$, then $\omega_g = f dx^1 \wedge ... \wedge dx^n$ for some smooth $f:U \subseteq M \rightarrow \R^n$. We will show that $f = \det(g_{ij})$.
    
    We have \textbf{explain the contraction}
    
    \begin{align*}
        f = C\Big( \omega_g, \frac{\pd}{\pd x^1} \wedge ... \wedge \frac{\pd}{\pd x^n} \Big) = C(\UUU^1 \wedge ... \wedge \UUU^n, \frac{\pd}{\pd x^1} \wedge ... \wedge \frac{\pd}{\pd x^n} \Big) = \det\Big( \UUU^j \Big( \frac{\pd}{\pd x^i} \Big) \Big).
    \end{align*}
    
    The last equality follows from the fact that the pushforward on the top exterior power is multiplication by the determinant. (Use Theorem \ref{ch::exterior_pwrs::rmk::top_pushforward_det}, where $\ff$ sends $\frac{\pd}{\pd x^i} \overset{\ff}{\mapsto} \UUU^j$).

    By Theorem \ref{ch::manifolds::thm::coords_frames_coframes} $[...] = \UUU^j \Big( \frac{\pd}{\pd x^i} \Big)$, so
    
    \begin{align*}
        f = \det \Big( \frac{\pd \UUU_j}{\pd x^i} \Big).
    \end{align*}
\end{proof}

\begin{theorem}
    \smallcite{book::SM}{389-390} (Riemannian volume form in local coordinates in terms of the Riemannian metric).

    Let $(M, g)$ be a smooth oriented Riemannian manifold, let $\{\UU_1, ..., \UU_n \}$ be an positively oriented orthonormal frame for $M$, and let $\{\UUU^1, ..., \UUU^n\}$ be the induced ordered orthonormal coframe. If $(U, \xx)$ is a smooth chart on $M$, then we have $\det \Big( \frac{\pd \UUU_j}{\pd x^i} \Big) = \sqrt{\det(g_{ij})}$, so it follows from the previous theorem that the Riemannian volume form is expressed on $U$ as

    \begin{align*}
        \omega_g = \sqrt{\det(g_{ij})} dx^1 \wedge ... \wedge dx^n.
    \end{align*}
\end{theorem}

\begin{proof}
    We show $\det \Big( \frac{\pd \UUU_j}{\pd x^i} \Big) = \sqrt{\det(g_{ij})}$. To do so, we compute $g_{ij} = g\Big( \frac{\pd}{\pd x^i}, \frac{\pd}{\pd x^j} \Big)$. Using the expression for $\frac{\pd}{\pd x^i}$ from the previous proof that was obtained by applying Theorem \ref{ch::manifolds::thm::change_coords_coord_frames}, we have
    
    \begin{align*}
        g_{ij} &= g\Big( \frac{\pd}{\pd x^i}, \frac{\pd}{\pd x^j} \Big)
        = g\Big( \sum_{k = 1}^n \frac{\pd \UUU_k}{\pd x^i} \frac{\pd}{\pd \UUU_k}, \sum_{\ell = 1}^n \frac{\pd \UUU_\ell}{\pd x^j} \frac{\pd}{\pd \UUU_\ell} \Big)
        = \sum_{k = 1}^n \sum_{\ell = 1}^n \frac{\pd \UUU_k}{\pd x^i} \frac{\pd \UUU_\ell}{\pd x^j} g\Big( \frac{\pd}{\pd \UUU_k}, \frac{\pd}{\pd \UUU_\ell} \Big) \\
        &= \sum_{k = 1}^n \sum_{\ell = 1}^n \frac{\pd \UUU_k}{\pd x^i} \frac{\pd \UUU_\ell}{\pd x^j} \delta^k_\ell
        = \sum_{k = 1}^n \frac{\pd \UUU_k}{\pd x^i} \frac{\pd \UUU_k}{\pd x^j}.
    \end{align*}
    
    (Note, we have $g\Big( \frac{\pd}{\pd \UUU_k}, \frac{\pd}{\pd \UUU_\ell} \Big) = \delta^k_\ell$ because the bases $\{\UU_1|_\pp, ..., \UU_n|_\pp\}$ and $\{\UUU^1|_\pp, ..., \UUU^n|_\pp\}$ are orthonormal).
    
    This last sum is the $ij$ entry of the matrix product $\Big( \frac{\pd \UUU_j}{\pd x^i} \Big)^\top \Big( \frac{\pd \UUU_j}{\pd x^i} \Big)$. Since $\det(\AA^\top) = \det(\AA)$, then $\det(g_{ij}) = \det \Big( \Big( \frac{\pd \UUU_j}{\pd x^i} \Big)^\top \Big( \frac{\pd \UUU_j}{\pd x^i} \Big) \Big) = \det\Big( \frac{\pd \UUU_j}{\pd x^i} \Big)^2$, which means $\det\Big( \frac{\pd \UUU_j}{\pd x^i} \Big) = \pm \sqrt{\det(g_{ij})}$. The sign of the square root must be positive because, since $\{\EE_1, ..., \EE_n\}$ is positively oriented, $\{\EEE^1, ..., \EEE^n\}$ [must be] as well.
\end{proof}