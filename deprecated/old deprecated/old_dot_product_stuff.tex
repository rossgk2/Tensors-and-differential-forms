
    \begin{lemma}
    \label{ch::lin_alg::lemma::polar}
        (Length of a projection).
        
        Let $\vv_1, \vv_2 \in \R^2$. Then the length of the projection of $\vv_1$ onto $\vv_2$ is $||\proj(\vv_1 \rightarrow \vv_2)|| = ||\vv_1||\cos(\theta)$, where $\theta$ is the unsigned angle between $\vv_1$ and $\vv_2$.
    \end{lemma}
    
    \begin{proof}
        The lemma holds in the special case when $\vv_2 = \see_1 = \begin{pmatrix} 1 \\ 0 \end{pmatrix}$; draw a right triangle to see this. For the general case, consider the rotation $\ff$ that satisfies $\ff(\hat{\vv}_2) = \see_1$, that is, $\ff(\vv_2) = ||\vv_2|| \see_1$. Then because rotations are length-preserving and with use of the previous lemma, $(v_1)_{||} = ||\proj(\vv_1 \rightarrow \vv_2)|| = ||\ff(\proj(\vv_1 \rightarrow \vv_2))|| = ||\proj(\ff(\vv_1) \rightarrow \ff(\vv_2))||$. This is the same as $||\proj(\ff(\vv_1) \rightarrow ||\vv_2|| \see_1)|| = ||\proj(\ff(\vv_1) \rightarrow \see_1)|| = ||\ff(\vv_1)||\cos(\phi)$, where $\phi$ is the unsigned angle between $\ff(\vv_1)$ and $\see_1$. We have $||\ff(\vv_1)|| = \vv_1$ because rotations are length-preserving, and $\phi = \theta$, where $\theta$ is the unsigned angle from $\vv_1$ to $\vv_2$, because rotations preserve angle. Therefore $||\proj(\vv_1 \rightarrow \vv_2)|| = ||\vv_1||\cos(\theta)$.
    \end{proof}
    
    
    \begin{theorem}
        (Geometric dot product in terms of unsigned angle).
        
        Since $\proj(\vv_1 \rightarrow \vv_2) = ||\vv_1||\cos(\theta)$, where $\theta$ is the unsigned angle between $\vv_1$ and $\vv_2$, the geometric dot product can also be written as
        
        \begin{align*}
            \vv_1 \cdot \vv_2 = ||\vv_1|| \spc ||\vv_2|| \cos(\theta).
        \end{align*}
    \end{theorem}

    \begin{remark}
        Vectors in $\R^2$ are perpendicular iff their geometric dot product is zero, since for $\vv_1, \vv_2 \neq \mathbf{0}$ we have $\vv_1 \cdot \vv_2 = 0 \iff \cos(\theta) = 0 \iff \theta = \pm \frac{\pi}{2}$.    
    \end{remark}
    
\section{The dot product on $\R^n$}

\begin{defn}
    (Dot product on $\R^n$). 
    
    Now that we have motivated the algebraic dot product on $\R^2$ by proving the previous theorem, we have a sensible way to define a dot product on $\R^n$. We don't do this by generalizing the geometric dot product on $\R^2$ because doing so requires a notion of ``angle in $\R^n$'', which is more complicated to set up than it's worth for our purposes\footnote{The unsigned angle between vectors $\vv_1, \vv_2 \in \R^n$ with $r = ||\vv_1|| = ||\vv_2||$ can be defined to be the ratio $\frac{\text{shortest path on the $n$-sphere of radius $r$ between the tips of $\vv_1$ and $\vv_2$}}{r}$.}. (After we define the dot product on $\R^n$, though, it will be relatively easy to define the unsigned angle between two vectors).
    
    Let $\sE$ be the standard basis of $\R^n$. We define $\cdot:\R^n \times \R^n \rightarrow \R$ by
    
    \begin{align*}
        \vv_1 \cdot \vv_2 := \sum_{i = 1}^n ([\vv_1]_\sE)_i ([\vv_2]_\sE)_i
    \end{align*}
    
    %Note that, for any orthonormal basis $\hat{U}$ of $\R^n$, we have
    
    %\begin{align*}
    %    \vv_1 \cdot \vv_2 := \sum_{i = 1}^n ([\vv_1]_{\hat{U}})_i ([\vv_2]_{\hat{U}})_i,
    %\end{align*}
    
    %because orthogonal linear functions on $\R^n$ preserve algebraic dot product (see Lemma \ref{ch::lin_alg::lemma::orthogonal_linear_fns_preserve_alg_dot_product}).
\end{defn}

\begin{theorem}
\label{ch::lin_alg::thm::dot_prod_matrix_matrix_prod}
    (Dot product on $\R^n$ as matrix-matrix product). 
    
    $\vv_1 \cdot \vv_2 = \vv_1^\top \vv_2$.
\end{theorem}

\begin{proof}
   Left as an exercise.
\end{proof}



\begin{defn}
\label{ch::lin_alg::defn::angle_in_Rn}
    (Unsigned angle in $\R^n$). 
    
    If $\ww_1, \ww_2 \in \R^2$, then since $\ww_1 \cdot \ww_2 = ||\ww_1|| \spc ||\ww_2||\cos(\theta)$, the unsigned angle between $\ww_1$ and $\ww_2$ is $\theta = \cos^{-1}\Big(\frac{\vv_1 \cdot \vv_2}{||\vv_1|| \spc ||\vv_2||}\Big)$. We define unsigned angle in $\R^n$ in analogy to this formula. The \textit{unsigned angle between vectors $\vv_1, \vv_2 \in \R^n$} is ${\theta := \cos^{-1}\Big(\frac{\vv_1 \cdot \vv_2}{||\vv_1|| \spc ||\vv_2||}\Big)}$, where the dot product here is the dot product on $\R^n$.
    
    Even in $\R^n$, the unsigned angle $\theta := \cos^{-1}\Big(\frac{\vv_1 \cdot \vv_2}{||\vv_1|| \spc ||\vv_2||}\Big)$ does indeed satisfy the condition $\theta \in [0, 2\pi)$. The proof of this is presented in a more general setting in Theorem \ref{ch::bilinear_forms_metric_tensors::thm::Cauchy_Schwarz}.
\end{defn}

\begin{remark}
\label{ch::lin_alg::rmk::geometric_dot_prod_Rn}
    (Geometric dot product on $\R^n$).
    
    With the previous definition of angle in $\R^n$, we have $\vv_1 \cdot \vv_2 = ||\vv_1|| \spc ||\vv_2||\cos(\theta)$ for $\vv_1, \vv_2 \in \R^n$, which looks like the formula for the geometric dot product on $\R^2.$
\end{remark}

\begin{defn}
\label{ch::lin_alg::defn::orthogonality_in_Rn}
    (Orthogonality of vectors in $\R^n$). 
    
    We say that vectors $\vv_1, \vv_2 \in \R^n$ are \textit{orthogonal} iff the angle in $\R^n$ between $\vv_1$ and $\vv_2$ is $\frac{\pi}{2}$. (So, orthogonality is a generalized notion of perpendicularity). Equivalently, $\vv_1, \vv_2 \in \R^n$ are orthogonal iff $\vv_1 \cdot \vv_2 = 0$.
\end{defn}

If you've been reading closely, you'll remember that we actually defined the dot product on $\R^2$ by using the notion of vector projection rather than by using the notion of an angle $\theta$. That is, in $\R^2$, the notion of vector projection lead to the notion of angle.
    
Since we have just defined ``unsigned angle in $\R^n$'', we can now define the notion of vector projection in $\R^n$. (So, we are taking the reverse of the approach that we did back in $\R^2$: the notion of angle- or, more specifically, of orthogonality- allows us to define vector projection).

\begin{defn}
    (Vector projection in $\R^n$).
    
    Consider vectors $\vv_1, \vv_2, (\vv_2)_\perp \in \R^n$, where $(\vv_2)_\perp$ is orthogonal to $\vv_2$. The \textit{vector projection} of $\vv_1$ onto $\vv_2$ is the unique vector $\proj(\vv_1 \rightarrow \vv_2) := (v_1)_{||} \hat{\vv}_2$ such that $\vv_1 = (v_1)_{||} \hat{\vv}_2 + (v_1)_\perp (\hat{\vv}_2)_\perp$, where $(v_1)_{||}, (v_1)_\perp \in K$. (This is, verbatim, the definition of vector projection in $\R^2$).
\end{defn}

\begin{theorem}
    (Dot product on $\R^n$ in terms of vector projection).
    
    Nicely enough, the dot product can be expressed in terms of a vector projection, just as it was in the old definition that we started with back in $\R^2$. The definition we started with in $\R^2$ holds in $\R^n$, verbatim:
    
    \begin{align*}
        \vv_1 \cdot \vv_2 =  || \proj(\vv_1 \rightarrow \vv_2) || \spc ||\vv_2||.
    \end{align*}
\end{theorem}

\begin{proof}
   We show that the above equation holds by determining $\proj$ in terms of the dot product. Consider $\vv_1, \vv_2, (\vv_2)_\perp \in \R^n$, where $(\vv_2)_\perp$ is orthogonal to $\vv_2$. We have $\vv_1 = (v_1)_{||} \vv_2 + (v_1)_\perp (\vv_2)_\perp$ for some $(v_1)_{||}, (v_1)_\perp \in \R$. Apply the map $\vv \mapsto \vv \cdot \vv_2$ to both sides to obtain that $\vv_1 \cdot \vv_2 = (v_1)_{||} \vv_2 \cdot \vv_2$. So $(v_1)_{||} = \frac{\vv_1 \cdot \vv_2}{\vv_2 \cdot \vv_2}$, which means $\proj(\vv_1 \rightarrow \vv_2) = \frac{\vv_1 \cdot \vv_2}{\vv_2 \cdot \vv_2} \vv_2 = \frac{\vv_1 \cdot \vv_2}{||\vv_2||^2}\vv_2 = \frac{\vv_1 \cdot \vv_2}{||\vv_2||}\hat{\vv}_2$. Thus, ${||\proj(\vv_1 \rightarrow \vv_2)|| \spc ||\vv_2|| = \frac{\vv_1 \cdot \vv_2}{||\vv_2||} ||\vv_2||}$, as claimed.
\end{proof}

\begin{theorem}
\label{ch::lin_alg::thm::vector_proj_dot_product}
    (Vector projection in terms of algebraic dot product).
    
    Let $\vv_1, \vv_2 \in \R^n$. We can express the projection of $\vv_1$ onto $\vv_2$ in terms of the dot product $\vv_1 \cdot \vv_2$. (Recall Definition \ref{ch::lin_alg::defn::vector_proj} for the definition of vector projection).
    
    Recall from Lemma \ref{ch::lin_alg::lemma::polar} that $||\proj(\vv_1 \rightarrow \vv_2)|| = ||\vv_1|| \cos(\theta)$. Since $\vv_1 \cdot \vv_2 = ||\vv_1|| \spc ||\vv_2|| \cos(\theta)$, then $||\proj(\vv_1 \rightarrow \vv_2)|| = \frac{\vv_1 \cdot \vv_2}{||\vv_2||}$. Thus
    
    \begin{align*}
        \proj(\vv_1 \rightarrow \vv_2) = \frac{\vv_1 \cdot \vv_2}{||\vv_2||} \hat{\vv}_2 = \frac{\vv_1 \cdot \vv_2}{\vv_2 \cdot \vv_2} \vv_2.
    \end{align*}
\end{theorem}

\begin{theorem}
\label{ch::lin_alg::thm::vector_proj_bilinear}
    (Vector projection in $\R^n$ is a bilinear function).
    
    Vector projection in $\R^n$ is a bilinear function. That is, $(\vv_1, \vv_2) \mapsto \proj(\vv_1 \rightarrow \vv_2)$ is linear in the argument $\vv_1$ when $\vv_2$ is fixed, and is linear in the argument $\vv_2$ when $\vv_1$ is fixed.
\end{theorem}

\begin{proof}
   This follows from the fact that vector projection in $\R^n$ can be expressed by using the dot product on $\R^n$, which is a bilinear function.
   
   In general, vector projection in any finite-dimensional vector space \textit{with inner product} is a bilinear function, because an inner product is by definition a bilinear function. We will discuss inner products and vector spaces with inner product in Chapter \ref{ch::bilinear_forms_metric_tensors}.
\end{proof}