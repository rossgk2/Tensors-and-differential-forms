\section{Orientation}

in inner product spaces over $\R$

\begin{defn}
    (Orientation in $\R^2$).

    An ordered basis $\{\ee_1, \ee_2\}$ of $\R^2$ is said to be \textit{counterclockwise-oriented} iff $|\theta_s(\ee_1, \ee_2)| < |\tttheta_s(\ee_1, \ee_2)|$ and \textit{clockwise-oriented} iff $|\theta_s(\ee_1, \ee_2)| > |\tttheta_s(\ee_1, \ee_2)|$.
\end{defn}

\begin{theorem}
    (Orientation in $\R^2$ in terms of signed projections).
    
    An ordered basis $\{\ee_1, \ee_2\}$ of $\R^2$ is counterclockwise-oriented iff $\sproj(\ee_2 \rightarrow \ee_1) > 0$ and clockwise oriented iff $\sproj(\ee_2 \rightarrow \ee_1) < 0$.
\end{theorem}

\begin{proof}
    [TO-DO]
\end{proof}

\begin{defn}
    (Orientation).

    Let $V$ be a $n$-dimensional inner product space over $\R$. The \textit{orientation} for $V$ induced by an ordered basis $E = (\ee_1, ..., \ee_n)$ for $V$ is the vector in $\{-1, 1\}^{n - 1}$ whose $i$th component is $\sgn(\sproj(\ee_{i + 1} \rightarrow \ee_i))$. We denote the orientation induced by an ordered basis $E$ by $\orn(E)$. Given ordered bases $E$ and $F$, we write $E \sim F$ iff $\orn(E) = \orn(F)$.

    In general, an \textit{orientation for $V$} is an element of $\{-1, 1\}^{\times(n - 1)}$.
\end{defn}

\begin{theorem}
    (Every orientation is induced by some ordered basis).

    Every ordered basis of a finite-dimensional inner product space over $\R$ is induced by some ordered basis of that inner product space.
\end{theorem}

\begin{proof}
    ...if need to change the $i$th sign of an orientation, negate the $(i + 1)$st vector...
\end{proof}

\begin{theorem}
    The orthonormalization algorithm\footnote{The algorithm from Theorem \ref{ch::bilinear_forms_metric_tensors::theorem::Gram-Schmidt}.} preserves induced orientation.
\end{theorem}

\begin{proof}

\end{proof}

The previous theorem allows us to study the orientation of arbitrary ordered bases by only considering the orientation of orthonormal ordered bases.

\subsection*{Orientation of orthonormal ordered bases}


\begin{defn}
    ($2$-dimensional rotation).

    Let $V$ be an $n$-dimensional inner product space, and let $\hU = (\huu_1, ..., \huu_n)$ be an orthonormal ordered basis for $V$. In analogy to the closed form for the standard matrix of a rotation in $\R^2$ (recall Theorem     \ref{ch::lin_alg::defn::standard_matrix_rotation}), we define
    
    \begin{align*}
        \RR_t &:=
        \kbordermatrix
        {
             & & & \text{$i$th column} &  & \text{$j$th column}   \\
             & 1 & \hdots & \cos(t) & 0 & -\sin(t) & 0 \\
             & 0 & & 0 & \vdots & 0 & \vdots \\
             & 0 & & \vdots & 1 & \vdots & \vdots \\
             & \vdots & & 0 & \vdots & 0 & \vdots \\
             & 0 & \hdots & \sin(t)  & 0 & \cos(t) & 1
        }, \\
        \widetilde{\RR}_t &:= \RR_{2\pi - t}.
    \end{align*}

    A \textit{$2$-dimensional rotation (in $V$)} is a linear function $\RR$ such that there exist $i, j$ for which the matrix $[\RR(\hU)]_{\hU}$ of $\RR$ relative to $\hU$ and $\hU$ is $\RR_t$. Note, since $\RR_t = \widetilde{\RR}_{2\pi - t}$, we can think of a $2$-dimensional rotation in $V$ as being either counterclockwise or clockwise.
\end{defn}

\begin{theorem}
    Every composition of $2$-dimensional rotations preserves induced orientation.
\end{theorem}

\begin{proof} 
    ...if there is such a composition of $2$-rotations, then that composition won't change the angles between vectors, and thus won't change the min of $|\theta_s(\ee_i, \ee_{i + 1})|$ and $|\tttheta_s(\ee_i, \ee_{i + 1})|$, let alone the argmin...
\end{proof}

\begin{lemma}
    In three dimensions, there is a composition of $2$-dimensional rotations taking any nonzero vector to any other vector.
\end{lemma}

\begin{proof}
    From old\_orientation.tex.
\end{proof}

\begin{lemma}
    There is composition of $2$-dimensional rotations taking any nonzero vector to any other vector.
\end{lemma}

\begin{proof}
    From old\_orientation.tex.
\end{proof}

\begin{theorem}
    For any orthonormal ordered basis there is a composition of $2$-dimensional rotations taking that basis to a permutation $\hU^\sigma$ of an orthonormal ordered basis $\hU$ with the standard orientation.
\end{theorem}

\begin{proof}
    From old\_orientation.tex.
\end{proof}

\subsection*{Orientation of permuted orthonormal ordered bases}

\begin{defn}
    (Permutation acting on a tuple). 
    
    Let $X = (x_1, ..., x_n)$ be an $n$-tuple. Given a permutation $\sigma \in S_n$, we define $X^\sigma := (x_{\sigma(1)}, ..., x_{\sigma(n)})$. For example, if $E = (\ee_1, ..., \ee_n)$ is a basis of a finite-dimensional vector space, then $E^\sigma = (\ee_{\sigma(1)}, ..., \ee_{\sigma(n)})$.
\end{defn}

\begin{theorem}
    \label{ch::lin_alg::thm::permutations_preserve_induced_orientation}

    (Permutations preserve induced orientation).
    
    If $E$ and $F$ are orthonormal ordered bases of an $n$-dimensional inner product space, then ${E \sim F \iff E^\sigma \sim F^\sigma}$ for all permutations $\sigma \in S_n$.
\end{theorem}

\begin{proof}
    [TO-DO]
\end{proof}

===============

\begin{theorem}
    A $2$-dimensional counterclockwise or clockwise rotation by $\pi/2$ takes an ordered basis to an ordered basis, inducing the same orientation as the original, that can be obtained from the original by swapping two vectors in the original and then negating one of them.
\end{theorem}

\begin{theorem}
    $\hU^\sigma \sim \hU^\pi \iff \sgn(\sigma) = \sgn(\pi) \text{ for all $\sigma, \pi \in S_n$}$
\end{theorem}

\begin{theorem}
    $\text{sorn}(\hU^\sigma) = \text{sgn}(\sigma)$
\end{theorem}

\subsection*{Orientation and the determinant}

\begin{theorem}
    The sign of the determinant of an ordered basis is its orientation.
\end{theorem}

\subsection*{Rotations in $\R^n$}

\begin{theorem}
    in $\R^2$, signed angle is equal to the product of orientation and unsigned angle.
\end{theorem}

\begin{defn}
    (Signed angle in $\R^n$).

    $\theta_s(\vv, \ww) := \orn((\vv, \ww)) \theta_u(\vv, \ww)$, where $\orn((\vv, \ww))$ is the orientation of $(\vv, \ww)$
\end{defn}

This definition allows for an easy generalization of the concept of rotation.

\begin{defn}
    (rotation in $\R^n$).

    A function $\R^n \rightarrow \R^n$ is a \textit{rotation in $\R^n$} if it fixes the origin, preserves length, and preserves signed angle in $\R^n$. Equivalently\footnote{Since length-preserving functions fix the origin iff they are linear.}, a function is a rotation in $\R^n$ iff it is linear, preserves length, and preserves orientation. 
\end{defn}

\begin{deriv}
    ($\{\text{rotations in $\R^n$}\} = \{ \text{linear functions on $\R^2$ that preserve length and orientation} \}$).

    A rotation in $\R^n$ is a linear function that preserves length and signed angle.

    If a function preserves length and signed angle then it also preserves signed projections, since the signed projection $\sproj(\vv \rightarrow \ww) = ||\vv|| \cos(\theta_u(\vv, \ww))$ is a function of length ($||\vv||$) and \textbf{not signed} angle \textbf{???} ($\cos(\theta_u(\vv, \ww))$). So every rotation is a linear function that preserves length and orientation.

    Is the converse true? Is is every linear function that preserves length and orientation a rotation? Well, suppose $\ff:\R^n \rightarrow \R^n$ is linear, preserves length, and preserves orientation. Then [reference Theorem about preserved quantities] it also preserves unsigned angle. Since $\ff$ preserves orientation and unsigned angle, and since signed angle is the product of orientation and signed angle, then $\ff$ preserves signed angle. Thus $\ff$ is linear, preserves length, and preserves signed angle; $\ff$ is a rotation. The converse is true!
\end{deriv}





