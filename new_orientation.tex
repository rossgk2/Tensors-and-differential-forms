\section{Orientation}

\begin{defn}
    (Orientation).

    Let $V$ be an $n$-dimensional inner product space. The \textit{orientation for $V$ induced by the ordered basis $E = (\ee_1, ..., \ee_n)$ for $V$} is the element of $\{-1, 1\}^{\times(n - 1)}$ whose $i$th component is $1$ iff $|\theta_s(\ee_i, \ee_{i + 1})| \leq |\tttheta_s(\ee_i, \ee_{i + 1})|$, and whose $i$th component is $-1$ iff $|\theta_s(\ee_i, \ee_{i + 1})| > |\tttheta_s(\ee_i, \ee_{i + 1})|$.
    
    We denote the orientation for $V$ induced by an ordered basis $E$ for $V$ by $\orn(E)$, and write $E \sim F$ iff $\orn(E) = \orn(F)$.
    
    In general, an \textit{orientation for $V$} is an element of $\{-1, 1\}^{\times(n - 1)}$.

    We define the \textit{standard orientation for $V$} to be $(1, 1, ..., 1) \in \{-1, 1\}^{\times(n - 1)}$.
\end{defn}

\begin{theorem}
    Every orientation is induced by some ordered basis.

    ..might not be true actually..
\end{theorem}

\begin{proof}
    Consider an orientation $o = (s_1, ..., s_n) \in \{-1, 1\}^{\times (n - 1)}$, of an $n$-dimensional inner product space. We must produce an ordered basis $E$ whose induced orientation is $o$.

    To do so, take any ordered basis $F = (\ff_1, ..., \ff_n)$ of the inner product space; it will have some orientation $(\widetilde{s}_1, ..., \widetilde{s}_n) \in \{-1, 1\}^{\times (n - 1)}$.
    
    Compare $o$ to $\orn(F)$ entry by entry. Whenever there is a difference, 
\end{proof}

\begin{theorem}
    The orthonormalization algorithm from Theorem \ref{ch::bilinear_forms_metric_tensors::theorem::Gram-Schmidt} preserves induced orientation.
\end{theorem}

\begin{proof}
    %In general, we have $\cos(\theta_u(\vv, \ww)) = \sproj(\vv \rightarrow \ww)/||\vv||$, and thus $\theta_u(\vv, \ww) = \arccos(\sproj(\vv \rightarrow \ww)/||\vv||)$. Therefore

    %$\theta_u(\ee_i, \ee_{i + 1}) = \arccos(\sproj(\ee_i \rightarrow \ee_{i + 1})/||\ee_i||)$

    %$\theta_u(\ee_{i + 1}, \ee_i) = \arccos(\sproj(\ee_{i + 1} \rightarrow \ee_i)/||\ee_{i + 1}||)$

    %The orthonormalization algorithm preserves the nonnegativity of ${\proj(\ee_{i + 1} \rightarrow \ee_i)}$ for all $i$. 
\end{proof}

\subsection*{Investigation into orientation via orthonormal ordered bases}

\begin{defn}
    ($2$-dimensional rotation).

    Let $V$ be an $n$-dimensional inner product space, and let $\hU = (\huu_1, ..., \huu_n)$ be an orthonormal ordered basis for $V$. In analogy to the closed form for the standard matrix of a rotation in $\R^2$ (recall Theorem     \ref{ch::lin_alg::defn::standard_matrix_rotation}), we define
    
    \begin{align*}
        \RR_t &:=
        \kbordermatrix
        {
             & & & \text{$i$th column} &  & \text{$j$th column}   \\
             & 1 & \hdots & \cos(t) & 0 & -\sin(t) & 0 \\
             & 0 & & 0 & \vdots & 0 & \vdots \\
             & 0 & & \vdots & 1 & \vdots & \vdots \\
             & \vdots & & 0 & \vdots & 0 & \vdots \\
             & 0 & \hdots & \sin(t)  & 0 & \cos(t) & 1
        }, \\
        \widetilde{\RR}_t &:= \RR_{2\pi - t}.
    \end{align*}

    A \textit{$2$-dimensional rotation (in $V$)} is a linear function $\RR$ such that there exist $i, j$ for which the matrix $[\RR(\hU)]_{\hU}$ of $\RR$ relative to $\hU$ and $\hU$ is $\RR_t$. Note, since $\RR_t = \widetilde{\RR}_{2\pi - t}$, we can think of a $2$-dimensional rotation in $V$ as being either counterclockwise or clockwise.
\end{defn}

\begin{theorem}
    Every composition of $2$-dimensional rotations preserves induced orientation.
\end{theorem}

\begin{proof} 
    ...if there is such a composition of $2$-rotations, then that composition won't change the angles between vectors, and thus won't change the min of $|\theta_s(\ee_i, \ee_{i + 1})|$ and $|\tttheta_s(\ee_i, \ee_{i + 1})|$, let alone the argmin...
\end{proof}

\begin{theorem}
    For any orthonormal ordered basis there is a composition of $2$-dimensional rotations taking that basis to a permutation $\hU^\sigma$ of an orthonormal ordered basis $\hU$ with the standard orientation.
\end{theorem}

\begin{theorem}
    (

    \begin{enumerate}
        \item A $2$-dimensional counterclockwise or clockwise rotation by $\pi/2$ takes an ordered basis to an ordered basis, inducing the same orientation as the original, that can be obtained from the original by swapping two vectors in the original and then negating one of them.

        \item $\hU^\sigma \sim \hU^\pi \iff \sgn(\sigma) = \sgn(\pi) \text{ for all $\sigma, \pi \in S_n$}$
    \end{enumerate}    
\end{theorem}

\begin{theorem}
    $\text{sorn}(\hU^\sigma) = \text{sgn}(\sigma)$
\end{theorem}


\begin{theorem}
    The sign of the determinant of an ordered basis is its orientation.
\end{theorem}

\subsection*{Rotations in $\R^n$}

\begin{theorem}
    signed angle in $\R^2$ is equal to the product of orientation and unsigned angle.
\end{theorem}

\begin{defn}
    (Signed angle in $\R^n$).

    $\theta_s(\vv, \ww) := \orn((\vv, \ww)) \theta_u(\vv, \ww)$, where $\orn((\vv, \ww))$ is the orientation of $(\vv, \ww)$
\end{defn}

This definition allows for an easy generalization of the concept of rotation.

\begin{defn}
    (rotation in $\R^n$).

    A function $\R^n \rightarrow \R^n$ is a \textit{rotation in $\R^n$} if it fixes the origin, preserves length, and preserves signed angle in $\R^n$. Equivalently\footnote{Since length-preserving functions fix the origin iff they are linear.}, a function is a rotation in $\R^n$ iff it is linear, preserves length, and preserves orientation. 
\end{defn}

\begin{deriv}
    ($\{\text{rotations in $\R^n$}\} = \{ \text{linear functions on $\R^2$ that preserve length and orientation} \}$).

    A rotation in $\R^n$ is a linear function that preserves length and signed angle.

    If a function preserves signed angle then it obviously preserves whether the smaller angle between two vectors is counterclockwise or clockwise, and therefore preserves orientation. So every rotation is a linear function that preserves length and orientation.

    Is the converse true? Is is every linear function that preserves length and orientation a rotation? Well, suppose $\ff:\R^n \rightarrow \R^n$ is linear,  preserves length, and preserves orientation. Then [reference Theorem about preserved quantities] it also preserves unsigned angle. Since $\ff$ preserves orientation and unsigned angle, and since signed angle is the product of orientation and signed angle, then $\ff$ preserves signed angle. Thus $\ff$ is linear, preserves length, and preserves signed angle; $\ff$ is a rotation. The converse is true!
\end{deriv}
