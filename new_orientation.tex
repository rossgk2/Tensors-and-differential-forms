\section{Orientation}

\begin{defn}
    (Orientation).
    
    Let $V$ be an $n$-dimensional inner product space. The \textit{orientation for $V$ induced by the ordered basis $E = (\ee_1, ..., \ee_n)$ for $V$} is the element of $\{-1, 1\}^{\times(n - 1)}$ whose $i$th component is $1$ whenever the counterclockwise angle from $\ee_i$ to $\ee_{i + 1}$ is less than or equal to than the corresponding clockwise angle, and whose $i$th component is $-1$ otherwise. We denote the orientation for $V$ induced by a basis $E$ for $V$ by $\iorn(E)$.
    
    In general, an \textit{orientation for $V$} is an element of $\{-1, 1\}^{\times(n - 1)}$.

    We define the \textit{standard orientation for $V$} to be $(1, 1, ..., 1) \in \{-1, 1\}^{n - 1}$.
\end{defn}

\begin{defn}
    ($2$-dimensional rotation).

    Let $V$ be an $n$-dimensional inner product space, and let $\hU = (\huu_1, ..., \huu_n)$ be an orthonormal ordered basis for $V$. In analogy to the closed form for the standard matrix of a rotation in $\R^2$ (recall Theorem     \ref{ch::lin_alg::defn::standard_matrix_rotation}), we define
    
    \begin{align*}
        \RR_t &:=
        \kbordermatrix
        {
             & & & \text{$i$th column} &  & \text{$j$th column}   \\
             & 1 & \hdots & \cos(t) & 0 & -\sin(t) & 0 \\
             & 0 & & 0 & \vdots & 0 & \vdots \\
             & 0 & & \vdots & 1 & \vdots & \vdots \\
             & \vdots & & 0 & \vdots & 0 & \vdots \\
             & 0 & \hdots & \sin(t)  & 0 & \cos(t) & 1
        }, \\
        \widetilde{\RR}_t &:= \RR_{2\pi - t}.
    \end{align*}

    A \textit{$2$-dimensional rotation (in $V$)} is a linear function $\RR$ such that there exist $i, j$ for which the matrix $[\RR(\hU)]_{\hU}$ of $\RR$ relative to $\hU$ and $\hU$ is $\RR_t$. Note, since $\RR_t = \RR_{2\pi - t}$, we can think of a $2$-dimensional rotation in $V$ as being either counterclockwise or clockwise.
\end{defn}

\begin{deriv}
    (Antisymmetry of induced orientation).

    Let $V$ be an $n$-dimensional inner product space. Hopefully, it is intuitive that counterclockwise $2$-dimensional rotations by $\pi/2$ and clockwise $2$-dimensional rotations by $\pi/2$ preserve induced orientation. (We prove this below, of course). Such rotations correspond to swap-and-negating vectors in ordered bases. Therefore, swap-and-negating vectors in a standard basis preserves induced orientation. More explicitly, for any ordered basis $E = (\ee_1, ..., \ee_n)$ of an $n$-dimensional inner product space, we have

    \begin{align*}
        \iorn((\ee_1, ..., \ee_i, ..., \ee_j, ..., \ee_n))
        =
        \iorn((\ee_1, ..., -\ee_j, ..., \ee_i, ..., \ee_n))
        =
        \iorn((\ee_1, ..., \ee_j, ..., -\ee_i, ..., \ee_n)) \text{ for all $i, j \in \{1, ..., n\}$}.
    \end{align*}
    
    That is, swapping and negating vectors in an ordered basis preserves the induced orientation.
\end{deriv}

\begin{proof}
    We need to prove that $\RR_{\pi/2}$ and $\widetilde{\RR}_{\pi/2}$ preserve induced orientation, i.e., that $\iorn(\RR_{\pi/2}(E)) = \iorn(E)$ and $\iorn(\widetilde{\RR}_{\pi/2}(E)) = \iorn(E)$ for any ordered basis $E$ of $V$. We show $\iorn(\RR_{\pi/2}(E)) = \iorn(E)$; the proof of the second fact is similar.

    [TO-DO]
\end{proof}

\begin{theorem}
    \label{ch::lin_alg::thm::antisymmetry_ordered_bases_signs}

    (Ordered bases of signed vectors).
    
    Let $E = (\ee_1, ..., \ee_n)$ be an ordered basis of an $n$-dimensional inner product space. If $s_1, ..., s_n \in \{-1, 1\}$, then

    \begin{align*}
        &\iorn((s_1 \ee_1, ..., s_n \ee_n)) = \\
        &\begin{cases}
            \iorn((\ee_1, ..., \ee_n)), & \text{the number of $s_i$ equal to $-1$ is $-1$ is even} \\
            \iorn((\ee_1, ..., \ee_{j - 1}, -\ee_j, \ee_{j + 1}, ..., \ee_n)) \text{ for any $j \in \{1, ..., n\}$}, & \text{the number of $s_i$ equal to $-1$ is $-1$ is odd} 
        \end{cases}.
    \end{align*}
\end{theorem}

\begin{proof}
    \mbox{} \\
    
    (Case: the number of $s_i$ equal to $-1$ is even). Since the number of $s_i$ equal to $-1$ is even, we can pair every $s_i \ee_i$ with $s_i = -1$ to some $s_j \ee_j$ with $s_j = -1$, so that all $s_i \ee_i$ with $s_i = -1$ appear as a member of only one pair. If we swap the vectors in each pair twice, then, as illustrated by the example $\iorn((-\ee_i, -\ee_j)) = \iorn((\ee_j, -\ee_i)) = \iorn((\ee_i, \ee_j))$, the coefficients on these vectors in the rightmost ordered basis will be $1$, with the order of the vectors being the same as in the original pair. Thus $(s_1 \ee_1, ..., s_n \ee_n)$ induces the same orientation as does $(\ee_1, ..., \ee_n)$.

    (Case: the number of $s_i$ equal to $-1$ is odd). Since the number of $s_i$ equal to $-1$ is odd, we can pair every $s_i \ee_i$ with $s_i = -1$ to some $s_j \ee_j$ with $s_j = -1$, so that all but one $s_i \ee_i$ with $s_i = -1$ appear as a member of only one pair. Let $s_k \ee_k$ be the vector with $s_k = -1$ that is not a member of a pair. Applying the same reasoning as in the previous case, we see that $(s_1 \ee_1, ..., s_n \ee_n)$ induces the same orientation as does $(\ee_1, ..., \ee_{k - 1}, -\ee_k, \ee_{k + 1}, ... \ee_n)$. Making use of the example $\iorn((-\ee_k, \ee_\ell)) = \iorn((\ee_\ell, \ee_k)) = \iorn((\ee_k, -\ee_\ell))$, we see that this ordered basis induces the same orientation as does $(\ee_1, \ee_2, ..., \ee_{\ell - 1}, -\ee_\ell, \ee_{\ell + 1}, ..., \ee_n)$ for any $\ell \in \{1, ..., n\}$. Thus, when the number of $s_i$ equal to $-1$ is odd, the ordered basis $(s_1 \ee_1, ..., s_n \ee_n)$ induces the same orientation as does $(\ee_1, ..., \ee_{\ell - 1}, -\ee_\ell, \ee_{\ell + 1}, ..., \ee_n)$ for any $\ell \in \{1, ..., n\}$. Replace $\ell$ with $j$ to obtain the result.
\end{proof}

\begin{defn}
    (Permutation acting on a tuple). 
    
    Let $X = (x_1, ..., x_n)$ be an $n$-tuple. Given a permutation $\sigma \in S_n$, we define $X^\sigma := (x_{\sigma(1)}, ..., x_{\sigma(n)})$. For example, if $E = (\ee_1, ..., \ee_n)$ is an ordered basis of a finite-dimensional vector space, then $E^\sigma = (\ee_{\sigma(1)}, ..., \ee_{\sigma(n)})$.
\end{defn}

\begin{theorem}
    \label{ch::lin_alg::thm::permutations_preserve_induced_orientation}

    (Permutations preserve induced orientation).
    
    If $E$ and $F$ are ordered bases of an $n$-dimensional inner product space, then ${\iorn(E) = \iorn(F) \iff \iorn(E^\sigma) = \iorn(F^\sigma)}$ for all permutations $\sigma \in S_n$.
\end{theorem}

\begin{proof}
    Let $E = (\ee_1, ..., \ee_n)$ and $F = (\ff_1, ..., \ff_n)$. [TO-DO]
\end{proof}

\begin{lemma}
    Let $E$ be an ordered basis of an $n$-dimensional inner product space. We have $\sgn(\sigma) = 1 \implies \iorn(E^\sigma) = \iorn(E)$ for all $\sigma \in S_n$.
\end{lemma}

\begin{proof}
    Let $E = (\ee_1, ..., \ee_n)$. The sign of a permutation is $1$ iff the permutation consists of an even number of swaps. So, an equivalent statement to the theorem is: for all $n$, if $\sigma$ consists of $2n$ transpositions, then $\iorn(E^\sigma) = \iorn(E)$. We prove this statement by induction on $n$.
    
    (Base case). Consider $E^\sigma$, where $\sigma = \tau \circ \pi$, and where $\pi, \tau$ are transpositions. 
    
    The permutation $\pi$ swaps some $i, j$. Using the example $\iorn((\ee_j, \ee_i)) = \iorn((-\ee_i, \ee_j))$, we see \\ $\iorn(E^\pi) = \iorn((\ee_1, ..., \ee_{i - 1}, -\ee_i, \ee_{i + 1}, ..., \ee_n))$. Since this ordered basis has only one basis vector with a coefficient of $-1$, it follows from the previous theorem that it induces the same orientation as does every other such ordered basis; in particular, it induces the same orientation as does $\hW :=  (-\ee_1, ..., \ee_n)$.

    So far, we've shown $\iorn(E^\pi) = \iorn(\hW)$. Since permutations preserve rotational equivalence (see Theorem \ref{ch::lin_alg::thm::permutations_preserve_induced_orientation}), we may apply the permutation $\tau$ to each side of the equation $\iorn(E^\pi) = \iorn(\hW)$ and use the fact $(E^\pi)^\tau = E^{\tau \circ \pi} = E^\sigma$ to obtain $\iorn(E^\sigma) = \iorn(\hW^\tau)$.

    We now compute the right side of this equation, $\iorn(\hW^\tau)$.

    \fullindent
    {
        (Case: $\tau$ swaps $1$ with some $\ell > 1$). By studying the example $\iorn((\ee_\ell, -\ee_1)) = \iorn((\ee_1, \ee_\ell))$, we see $\iorn(\hW) = \iorn((-\ee_1, ..., \ee_n)^\tau) = \iorn ((\ee_1, ..., \ee_n)) = \iorn(E)$.
    }
    \fullindent
    {
        (Case: $\tau$ swaps $k \neq 1$ with some $\ell > k$). By studying the example $\iorn((\ee_\ell, \ee_k)) = \iorn((-\ee_k, \ee_\ell))$, we see $\iorn(\hW^\tau) = \iorn((-\ee_1, ..., \ee_n)^\tau) = \iorn((-\ee_1, ..., \ee_{k - 1}, -\ee_k, \ee_{k + 1}, ..., \ee_n))$. In this last ordered basis, the number of basis vectors with a coefficient of $-1$ is even. It follows from the previous theorem that this last ordered basis induces the same orientation as does $E$.
    }

    We've shown that $\iorn(\hW^\tau) = \iorn(E)$ for all $\tau \in S_n$. Therefore, we have $\iorn(E^\sigma) = \iorn(\hW^\tau) = \iorn(E)$, as desired.
    
    (Inductive case). Assume as the inductive hypothesis if $\sigma$ consists of $2n$ transpositions, then $\iorn(E^\sigma) = \iorn(E)$. We need to prove that if $\sigma$ consists of $2(n + 1)$ transpositions, then $\iorn(E^\sigma) = \iorn(E)$.

    So, assume $\sigma$ consists of $2(n + 1) = 2n + 2$ transpositions. Then $\sigma = \tau \circ \pi$, where $\pi$ consists of $2n$ transpositions and $\tau$ consists of $2$ transpositions. By the inductive hypothesis, $\iorn(E^\pi) = \iorn(E)$. Since permutations preserve induced orientation (see Theorem \ref{ch::lin_alg::thm::permutations_preserve_induced_orientation}), and since $(E^\pi)^\tau = E^{\tau \circ \pi} = E^\sigma$, we have $\iorn(E^\sigma) = \iorn(E^\tau)$. Since $\tau$ consists of $2$ transpositions, then from the base case it follows that $\iorn(E^\tau) = \iorn(E)$. Thus $\iorn(E^\sigma) = \iorn (E^\tau) = \iorn(E)$, as desired.
\end{proof}

\begin{theorem}
    \begin{align*}
        E^\sigma \sim E^\pi \iff \sgn(\sigma) = \sgn(\pi) \text{ for all $\sigma, \pi \in S_n$}.
    \end{align*}
\end{theorem}

\begin{proof}
    \mbox{} \\ \indent
    ($\impliedby$).
    
    The reverse implication is true when the following two statements are:
    
    \begin{enumerate}
        \item If $\sgn(\pi) = 1$, then $\sgn(\sigma) = 1 \implies \iorn(E^\sigma) = \iorn(E^\pi)$ for all $\sigma \in S_n$.
        \item If $\sgn(\pi) = -1$, then $\sgn(\sigma) = -1 \implies \iorn(E^\sigma) = \iorn(E^\pi)$ for all $\sigma \in S_n$.
    \end{enumerate}

    We restate the previous lemma for convenience:

    \begin{itemize}
        \item $\sgn(\sigma) = 1 \implies \iorn(E^\sigma) = \iorn(E)$ for all $\sigma \in S_n$.
    \end{itemize}

    Now we prove (1) and (2).

    \begin{enumerate}
        \item If $\sgn(\sigma) = \sgn(\pi) = 1$, then $\iorn(E^\sigma) = \iorn(E) = \iorn (E^\pi)$ by the above bullet.
        \item Let $\pi$ be any odd permutation. If $\sigma$ is an even permutation, then applying the above bullet yields $\iorn((E^\pi)^\sigma) = \iorn(E^\pi)$, i.e. $\iorn(E^{\sigma \circ \pi}) = \iorn(E^\pi)$. Thus $\sgn(\sigma) = 1 \implies \iorn(E^{\sigma \circ \pi}) = \iorn(E^\pi)$. In general, every odd permutation is of the form $\sigma \circ \pi$, where $\sigma$ is some even permutation. Thus, as $\sigma$ varies over the even permutations, the permutation $\tau := \sigma \circ \pi$ varies over the odd permutations. This makes the statement $\sgn(\sigma) = 1 \implies \iorn(E^{\sigma \circ \pi}) = \iorn(E^\pi)$ equivalent to the statement $\sgn(\tau) = -1 \implies \iorn(E^\tau) = \iorn(E^\pi)$. Since the former statement is true, so is the later, as desired.
    \end{enumerate}

   ($\implies$).

    The forward implication is true when the following two statements are:

    \begin{enumerate}
        \item[3.] If $\sgn(\pi) = 1$, then $\iorn(E^\sigma) = \iorn(E^\pi) \implies \sgn(\sigma) = 1$ for all $\sigma \in S_n$.
        \item[4.] If $\sgn(\pi) = -1$, then $\iorn(E^\sigma) = \iorn(E^\pi) \implies \sgn(\sigma) = -1$ for all $\sigma \in S_n$.
    \end{enumerate}

    Before we can prove (3) and (4), we need to prove the following bullet point:

    \begin{itemize}
        \item $\iorn(E) = \iorn(E^\sigma) \implies \sgn(\sigma) = 1$ for all $\sigma \in S_n$.

        We show the contrapositive, $\sgn(\sigma) = -1 \implies \iorn(E) \neq \iorn(E^\sigma)$ for all $\sigma \in S_n$. Because of (2), it suffices to show that $\iorn(E) \nsim \iorn(E^\sigma)$ for \textit{any} $\sigma$ with $\sgn(\sigma) = -1$.

        Since $\sgn(\transp{1}{2}) = -1$, we will show that $\iorn(E) \neq \iorn(E^{(1 \spc 2)})$.
            
        Suppose for contradiction that $\iorn(E) = \iorn(E^{(1 \spc 2)})$, i.e., that $\iorn((\ee_1, \ee_2, \ee_3 ..., \ee_n)) = \iorn((\ee_2, \ee_1, \ee_3 ..., \ee_n))$. [TO-DO; write new contradiction proof]

        % Old contradiction proof, from when I thought orientation should be defined only in terms of rotational equivalence (which is of course wrong):
        
        %Then there exists $t \in [0, 2\pi)$ such that ${\RR_t(E) = (\RR_t(\ee_1), \RR_t(\ee_2), \RR_t(\ee_3), ..., \RR_t(\ee_n)) = (\ee_2, \ee_1, \ee_3, ..., \ee_n) = E^{(1 \spc 2)}}$. From the first two entries of these tuples, we have $[\RR_t(\ee_1)]_{E} = [\ee_2]_{E} = (0, 1)^\top$ and $[\RR_t(\ee_2)]_{E} = [\ee_1]_{E} = (1, 0)^\top$.
            
        % Since $[\RR_t(\ee_1)]_{E} = (\cos(t), \sin(t))^\top$ and ${[\RR_t(\ee_2)]_{E} = (-\sin(t), \cos(t))^\top}$, then this is equivalent to $((\cos(t), \sin(t))^\top = (0, 1)^\top \text{ and } (-\sin(t), \cos(t))^\top = (1, 0)^\top)$, which further implies $\sin(t) = 1$ and $\sin(t) = -1$. Since $\sin$ is a well-defined function, this is a contradiction.
    \end{itemize}

    Now we prove (3) and (4).
    
    \begin{enumerate}
        \item[3.] Let $\pi$ be any even permutation. If $\sigma$ is an even permutation, then applying the above bullet yields $\iorn((E^\pi)^\sigma) = \iorn(E^\pi) \implies \sgn(\sigma) = 1$, i.e., $\iorn(E^{\sigma \circ \pi}) = \iorn(E^\pi) \implies \sgn(\sigma) = 1$. If we allow $\pi$ to vary over the even permutations, then since $\sigma$ varies over the even permutations, $\tau := \sigma \circ \pi$ varies over the even permutations, and we obtain the statement $\iorn(E^\tau) = \iorn(E^\pi) \implies \sgn(\tau) = 1$ for all $\tau \in S_n$, as desired.
        \item[4.] The proof is analogous to that of (3). Begin the proof with ``Let $\pi$ be any odd permutation'', and use the fact that since $\sigma$ varies over the even permutations, $\tau := \sigma \circ \pi$ varies over the odd permutations.
    \end{enumerate}
\end{proof}

\begin{theorem}
    (Orientation and parity).

    \begin{enumerate}

        \item Since parities of permutations can only take on two values, there are only two orientations. [\textit{positive} orientation and \textit{negative} orientation]
        
        \item (Identification of orientation with sign). We define $\orn(E)$ to be $1$ when $E$ induces positive orientation and $-1$ when $E$ induces negative orientation.
    \end{enumerate}
\end{theorem}

\begin{proof}

\end{proof}

\begin{theorem}    
    $\orn(\hU^\sigma) = \sgn(\sigma)$.
\end{theorem}

\begin{theorem}
    (Taking an arbitrary orthonormal ordered basis to a permutation of a chosen orthonormal ordered basis).
    
    For every chosen orthonormal ordered basis $\hU$ there exists a composition of 2-dimensional rotations that takes an arbitrary orthonormal ordered basis $\hW$ to a permutation $\hU^\sigma$ of the chosen orthonormal ordered basis.
\end{theorem}

\begin{theorem}
    (Taking an arbitrary ordered basis to an orthonormal ordered basis).

    \begin{enumerate}
        \item Orthonormalization (Gram-Schmidt) preserves orientation.
        \item Orthonormalization (Gram-Schmidt) takes an arbitrary ordered basis to an orthonormal ordered basis.
    \end{enumerate}
\end{theorem}

\begin{proof}
     \mbox{} \\
    \begin{enumerate}
        \item Gram-Schmidt preserves the nonnegativity of $\proj(\vv \rightarrow \huu_i)$ for all $i$. Therefore it preserves orientation.
        \item This was shown in Theorem \ref{ch::bilinear_forms_metric_tensors::theorem::Gram-Schmidt}.
    \end{enumerate}
\end{proof}

\begin{theorem}
    The sign of the determinant of an ordered basis is equal to its orientation.
\end{theorem}

\begin{proof}
    take an arbitrary orthonormal ordered basis to a $\hU^\sigma$ for some $\sigma$, where $\hU$ is positively oriented. ...
\end{proof}

\subsection*{Rotations in $\R^n$}

\begin{theorem}
    signed angle in $\R^2$ is equal to the product of orientation and unsigned angle.
\end{theorem}

\begin{defn}
    (Signed angle in $\R^n$).

    $\theta_s(\vv, \ww) := \orn((\vv, \ww)) \theta_u(\vv, \ww)$, where $\orn((\vv, \ww))$ is the orientation of $(\vv, \ww)$
\end{defn}

This definition allows for an easy generalization of the concept of rotation.

\begin{defn}
    (rotation in $\R^n$).

    A function $\R^n \rightarrow \R^n$ is a \textit{rotation in $\R^n$} if it fixes the origin, preserves length, and preserves signed angle in $\R^n$. Equivalently\footnote{Since length-preserving functions fix the origin iff they are linear.}, a function is a rotation in $\R^n$ iff it is linear, preserves length, and preserves orientation. 
\end{defn}

\begin{deriv}
    ($\{\text{rotations in $\R^n$}\} = \{ \text{linear functions on $\R^2$ that preserve length and orientation} \}$).

    A rotation in $\R^n$ is a linear function that preserves length and signed angle.

    If a function preserves signed angle then it obviously preserves whether the smaller angle between two vectors is counterclockwise or clockwise, and therefore preserves orientation. So every rotation is a linear function that preserves length and orientation.

    Is the converse true? Is is every linear function that preserves length and orientation a rotation? Well, suppose $\ff:\R^n \rightarrow \R^n$ is linear,  preserves length, and preserves orientation. Then [reference Theorem about preserved quantities] it also preserves unsigned angle. Since $\ff$ preserves orientation and unsigned angle, and since signed angle is the product of orientation and signed angle, then $\ff$ preserves signed angle. Thus $\ff$ is linear, preserves length, and preserves signed angle; $\ff$ is a rotation. The converse is true!
\end{deriv}