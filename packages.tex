\usepackage[T2A, T1]{fontenc}
\usepackage[utf8]{inputenc}
\usepackage{lmodern}
\usepackage{hyperref}
\usepackage{graphicx}
\usepackage{float} %for fixing position of tables with [H]
\usepackage[english]{babel}
\usepackage{amsmath}
\usepackage{amssymb} %for bold Greek letters
\usepackage{amsfonts}
\usepackage{tikz-cd}
\usetikzlibrary{cd}
\usepackage{indentfirst} %indent the first line of each paragraph
\usepackage{comment} %enables the commenting-out of large blocks of text with \begin{comment} and \end{comment}
\usepackage{forloop}
\usepackage{hyperref} %clickable hyperlinks with \url{} 
\usepackage{empheq} %for boxed multi-line equations
\usepackage{aligned-overset} %so that using \overset{\text{helpful hint}}{=} doesn't mess up align environments
\usepackage{cancel} %for striking-through symbols
%https://trucastuces.wordpress.com/2012/10/10/boxed-equations-in-latex/
\usepackage{ulem} %for strikethrough in the todo list
\usepackage{scrextend} % so that \addmargin{} works

\usepackage{kbordermatrix} %for labeling rows/columns of matrices (http://www.hss.caltech.edu/~kcb/TeX/kbordermatrix.sty)
\renewcommand{\kbldelim}{(}% Left delimiter
\renewcommand{\kbrdelim}{)}% Right delimiter

%Bigger section, subsection, and subsubsection fonts
\makeatletter
\renewcommand\section{\@startsection {section}{1}{\z@}%
                                   {-3.5ex \@plus -1ex \@minus -.2ex}%
                                   {2.3ex \@plus.2ex}%
                                   {\normalfont\LARGE\bfseries}}% from \Large
\renewcommand\subsection{\@startsection{subsection}{2}{\z@}%
                                     {-3.25ex\@plus -1ex \@minus -.2ex}%
                                     {1.5ex \@plus .2ex}%
                                     {\normalfont\Large\bfseries}}% from \large
\renewcommand\subsubsection{\@startsection{subsubsection}{3}{\z@}%
                                     {-3.25ex\@plus -1ex \@minus -.2ex}%
                                     {1.5ex \@plus .2ex}%
                                     {\normalfont\large\bfseries}}% from \normalsize
\makeatother

\usepackage{amsthm} %for defns, lemmas, theorems, proofs
%https://tex.stackexchange.com/questions/12913/customizing-theorem-name
%https://tex.stackexchange.com/questions/161180/common-numbering-for-theorems-and-definitions

%Definitions
\newtheoremstyle{defn_}{}{}{}{}{\bfseries}{.}{.5em}{}
\theoremstyle{defn_}
\newtheorem{defn}{Definition}
\numberwithin{defn}{chapter}

%Derivations
\newtheoremstyle{deriv_}{}{}{}{}{\bfseries}{.}{.5em}{}
\theoremstyle{deriv_}
\newtheorem{deriv}[defn]{Derivation}

%Theorems
\newtheoremstyle{theorem_}{}{}{}{}{\bfseries}{.}{.5em}{}
\theoremstyle{theorem_}
\newtheorem{theorem}[defn]{Theorem}

%Lemmas
\newtheoremstyle{lemma_}{}{}{}{}{\bfseries}{.}{.5em}{}
\theoremstyle{lemma_}
\newtheorem{lemma}[defn]{Lemma}

%Corollaries
\newtheoremstyle{cor_}{}{}{}{}{\bfseries}{.}{.5em}{}
\theoremstyle{cor_}
\newtheorem{cor}[defn]{Corollary.}

%Remarks
\newtheoremstyle{remark_}{}{}{}{}{\bfseries}{.}{.5em}{}
\theoremstyle{remark_}
\newtheorem{remark}[defn]{Remark}

%Examples
\newtheoremstyle{example_}{}{}{}{}{\bfseries}{.}{.5em}{}
\theoremstyle{example_}
\newtheorem{example}[defn]{Example}

\usepackage{geometry}
\geometry
{
    a4paper,
    total={170mm,257mm},
    left=20mm,
    top=20mm,
}