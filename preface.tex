\chapter*{About this book}

This book has four primary goals:

\begin{enumerate}
    \item Give a comprehensive understanding of ``multilinear elements'', or \textit{tensors}, to an extent that enables future study of tensor-intensive physics topics (e.g. general relativity).
    \item Study multilinear and antisymmetric notions (the determinant, orientation, and the wedge product), and then learn how the wedge product is related to tensors.
    \item Generalize calculus to the setting of higher dimensional surfaces, or \textit{manifolds}, by using multilinear and antisymmetric notions.
    \item Use this calculus on manifolds to prove the generalized Stokes' theorem.
\end{enumerate}

\section*{Tentative prerequisites and reading advice for a student}

This book is primarily written for a reader who has experience with the following:

\begin{itemize}
    \item the content of typical three-course calculus sequence: single-variable differential calculus, single-variable integral calculus, and multivariable calculus (but \textit{not} differential equations)
    \item introductory linear algebra
    \item logic and proof writing
\end{itemize}

A dedicated reader who has only taken the three-course calculus sequence mentioned above can still understand everything in this book with a bit of extra effort; such a reader will have to do a thorough reading of the linear algebra chapter.

Every reader should take \textit{some} advantage of the linear algebra chapter, as the content from this chapter is applied almost constantly throughout this book. (The new ideas of tensors and differential forms are mostly in the realm of linear algebra- not calculus, as one might initially suspect).

My advice is to use the linear algebra chapter as a guide for learning the core theory and to consult an introductory linear algebra textbook, such as \textit{Linear Algebra Done Wrong}\footnote{Funnily enough, Treil's \textit{Linear Algebra Done Wrong} is better in many ways than Axler's \textit{Linear Algebra Done Right}. The former explains why matrix-vector products are the way they are; the later does not, and gives an opaque definition of the matrix of a linear function relative to bases.} by Sergei Treil (this is available for free online) or \textit{Linear and Geometric Algebra} by Alan Macdonald for concrete examples. Another text you might consult is Steven Roman's \textit{Advanced Linear Algebra}. Be warned: I have found no linear algebra book that \textit{truly} satisfactorily explains the matrix with respect to bases of a linear function, matrix-vector products, or matrix-matrix products; even theoretical treatments miss the mark by focusing on the fact that linear functions correspond to matrices (rather than focusing on why this correspondence happens). To be fair, \textit{Linear Algebra Done Wrong} does come very close to some of the best possible explanations. In any case, be wary when reading about these concepts in other books.